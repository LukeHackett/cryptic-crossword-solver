\section{Rapid Application Development}
The Rapid Application Development (RAD) model is an extension to the 
incremental development methodology. The RAD model states that all requirements
should be treated as mini projects, and that they should be completed in 
parallel. Each of the mini projects are ran like a normal project, and hence 
time scales need to be adhered to \citep{istqb10}.

Upon completion of the mini project, the customer is able to review the output,
and provide value feedback regarding to the delivery and the requirements. RAD 
will follow a somewhat simpler primary path, allowing for business modelling, 
data modelling, process modelling, application generation, testing and turnover
\citep{istqb10}.


\subsection{Advantages}
\citet{istqb10} states that there are many advantages of adopting the RAD model
within a team:

\begin{itemize}
  \item A reduced development time, due to the fact that the business modelling
  and data modelling processes should cover all aspects
  \item The combination of Data modelling and Process modelling should allow 
  for the increased ability to reuse components
  \item Reviews of delivered outputs are constantly reviewed by the customer, 
  allows for early feedback to be gained
  \item Parts of the system are integrated at an earlier stage, which allows 
  for fewer integration issues towards the end of the project
\end{itemize}


\subsection{Disadvantages}
However, \citet{istqb10} also states that there can be disadvantages of 
adopting the RAD model within a team:

\begin{itemize}
  \item There is a high dependency upon an overall strong team and strong 
  individual performances for identifying business requirements
  \item The model will only work for systems that can be modularised
  \item The model assumes that the team members are highly skills designers and
  developers, with an even higher dependency upon modelling skills
\end{itemize}
