These following requirements define the functional needs of the product to allow a means to return a result upon given a clue.

\noindent\llap{\textbf{[R8/1]}}The system shall take an input \textbf{\textbf{(clue)}} and output \textbf{(cluetype)} after it has been determined whether it is:

\begin{enumerate} [(a)] %Alphabetical list
\item Purely Cryptic 				\hfill{[R9/1]}
\item Hidden 						\hfill{[R10/1]}
\item Charades 						\hfill{[R11/1]}
\item Anagrams 						\hfill{[R12/1]}
\item Homophones 					\hfill{[R13/1]}
\item Acrostics 					\hfill{[R14/1]}
\item Palindromes 					\hfill{[R15/1]}
\item Reversals 					\hfill{[R16/1]}
\item \& lit 						\hfill{[R17/1]}
\item Double Definition 			\hfill{[R18/1]}
\item Containers 					\hfill{[R19/1]}
\item Deletions 					\hfill{[R20/1]}
\item Spoonerisms 					\hfill{[R21/1]}
\item Pattern 						\hfill{[R22/1]}
\item Substitutions 				\hfill{[R23/1]}
\item Shifting 						\hfill{[R24/1]}
\item Exchange 						\hfill{[R25/1]}
\end{enumerate}

\textbf{Rationale:}  There are 17 different types of clues, the solver needs to be able to solve each type of clue\\
\textbf{Volatility:} High

\noindent\llap{\textbf{[R9/1]}}The system shall take an input \textbf{(clue)} and output \textbf{(pure)} if the \textbf{(clue)} contains a Question mark \textbf{(?)} and/or an Exclamation mark \textbf{(!)}\\

\textbf{Rationale:}  \\
\textbf{Volatility:} 

\noindent\llap{\textbf{[R10/1]}}The system shall take an input \textbf{(clue)} and output (hidden) if the \textbf{(clue)}

\textbf{Rationale:}  \\
\textbf{Volatility:} 

\noindent\llap{\textbf{[R11/1]}}The system shall take an input \textbf{(clue)} and output (charade) if the \textbf{(clue)}

\textbf{Rationale:}  \\
\textbf{Volatility:} 

\noindent\llap{\textbf{[R12/1]}}The system shall take an input \textbf{(clue)} and output (anagram) if the \textbf{(clue)}

\textbf{Rationale:}  \\
\textbf{Volatility:} 

\noindent\llap{\textbf{[R13/1]}}The system shall take an input \textbf{(clue)} and output (homophone) if the \textbf{(clue)}

\textbf{Rationale:}  \\
\textbf{Volatility:} 

\noindent\llap{\textbf{[R14/1]}}The system shall take an input \textbf{(clue)} and output (accrostic) if the \textbf{(clue)}

\textbf{Rationale:}  \\
\textbf{Volatility:} 

\noindent\llap{\textbf{[R15/1]}}The system shall take an input \textbf{(clue)} and output (palindrom) if the \textbf{(clue)}

\textbf{Rationale:}  \\
\textbf{Volatility:} 

\noindent\llap{\textbf{[R16/1]}}The system shall take an input \textbf{(clue)} and output (reversal) if the \textbf{(clue)}

\textbf{Rationale:}  \\
\textbf{Volatility:} 

\noindent\llap{\textbf{[R17/1]}}The system shall take an input \textbf{(clue)} and output (lit) if the \textbf{(clue)}

\textbf{Rationale:}  \\
\textbf{Volatility:} 

\noindent\llap{\textbf{[R18/1]}}The system shall take an input \textbf{(clue)} and output (double\_def) if the \textbf{(clue)}

\textbf{Rationale:}  \\
\textbf{Volatility:} 

\noindent\llap{\textbf{[R19/1]}}The system shall take an input \textbf{(clue)} and output (container) if the \textbf{(clue)}

\textbf{Rationale:}  \\
\textbf{Volatility:} 

\noindent\llap{\textbf{[R20/1]}}The system shall take an input \textbf{(clue)} and output (deletion) if the \textbf{(clue)}

\textbf{Rationale:}  \\
\textbf{Volatility:} 

\noindent\llap{\textbf{[R21/1]}}The system shall take an input \textbf{(clue)} and output (spoonerism) if the \textbf{(clue)}

\textbf{Rationale:}  \\
\textbf{Volatility:} 

\noindent\llap{\textbf{[R22/1]}}The system shall take an input \textbf{(clue)} and output (pattern) if the \textbf{(clue)}

\textbf{Rationale:}  \\
\textbf{Volatility:} 

\noindent\llap{\textbf{[R23/1]}}The system shall take an input \textbf{(clue)} and output (substitution) if the \textbf{(clue)}

\textbf{Rationale:}  \\
\textbf{Volatility:} 

\noindent\llap{\textbf{[R24/1]}}The system shall take an input \textbf{(clue)} and output (shifting) if the \textbf{(clue)}

\textbf{Rationale:}  \\
\textbf{Volatility:} 

\noindent\llap{\textbf{[R25/1]}}The system shall take an input \textbf{(clue)} and output (exchange) if the \textbf{(clue)}

\textbf{Rationale:}  \\
\textbf{Volatility:} 