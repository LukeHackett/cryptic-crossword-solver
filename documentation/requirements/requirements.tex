\chapter{Software Specification}
\label{chap:requirements}
\citet{cadle10} states that there is a standard hierarchical approach in 
structuring requirements. Table \ref{table:requirementsCategories} outlines
the main four categories:

\begin{table}[H]
  \begin{tabular}{|l|l|l|l|}
    \hline
    {\bf General} & {\bf Technical} & {\bf Functional} & {\bf Non-Functional} \\ 
    \hline
    Business constraints & Hardware & Data entry & Performance \\ 
    Business policies & Software & Data maintenance & Security \\ 
    Legal & Interoperability & Procedure & Legal and Access \\ 
    Branding & Internet & Retrieval & Backup and Recovery \\ 
    Cultural & ~ & ~ & Archiving and Retention \\ 
    Language & ~ & ~ & Maintainability \\ 
    ~ & ~ & ~ & Business Continuity \\ 
    ~ & ~ & ~ & Availability \\ 
    ~ & ~ & ~ & Usability \\ 
    ~ & ~ & ~ & Capacity \\
    \hline
  \end{tabular}
  \label{table:requirementsCategories}
\end{table}

Each of the above categories will form the basis of this requirements chapter,
along with any additional user requirements, assumptions and a risk assessment.

% Project Drivers
\newpage
\section{Purpose}

\subsection{Background of the Project}

The purpose of this project is to produce an app that will be compatible with the three main mobile operating systems, iOS, Android and Blackberry OS. The app is to be able to solve given clues from cryptic crosswords which are widely available on publications such as the Burgundian newspaper. There is no current form mobile application in the current market which solves cryptic clues. The produced product is to allow the end user to solve some if not all types of clues which have been discussed in section 2.1.3.

\subsection{Project Goals}

The main goals of the project are:

\begin{enumerate}
  \item Be able to solve a given clue.
  \item Identify the type of clue it is.
  \item Show a stack trace of how the clue was deduced.
  \item Be able to store previous solved clues. 
\end{enumerate}


% Client, customer and stakeholder
\newpage
\section{The Client, the Customer and Other Stakeholders}

\subsection{The Client}

 % Client
	\textbf{Dr Hugh \textsc{Osborne}}\\
	Senior Lecturer\\
	University Of Huddersfield\\
	h.r.osborne@hud.ac.uk

\subsection{The Customer}

The intended customer of the product are users of smartphone and tablets whom are looking to solve all those unsolvable Cryptic Crosswords. The applications will be deployed on the app market for the three listed mobile operating systems which means that the app will be available to anyone who has a compatible device with the required software. The physical deployment of the application is out of the project scope so a price for the deployment will not be discussed.

\subsection{Other Stakeholders}

For the purpose of the project the other stakeholders are as follows:

 % Supervisor
      \emph{Project Supervisor}\\
      \textbf{Dr. Gary \textsc{Allen}} \\
      Senior Lecturer \\
      University Of Huddersfield \\
      g.allen@hud.ac.uk 

  % Examiner
      \emph{Project Examiner:} \\ 
      \textbf{Sotirios \textsc{Batsakis}}\\
      University Of Huddersfield\\
      s.batsakis-STA@unimail.hud.ac.uk

  % Moderator
      \emph{Internal Moderator}\\
      \textbf{Collin \textsc{Venters}} \\
      Senior Lecturer \\
      University Of Huddersfield \\
      c.venters@hud.ac.uk



% General Requirements
\newpage
\section{General Requirements}

\subsection{Business constraints}

\subsection{Business policies}

\subsection{Legal}

\subsection{Branding}

\subsection{Cultural}

\subsection{Language}



% Technical Requirements
\newpage
\section{Technical Requirements}

\subsection{Hardware}

\subsection{Software}

\subsection{Interoperability}

\subsection{Internet}



% Functional Requirements
\newpage
\section{Functional Requirements}

These are things the product must do.

e.g The product shall produce a result upon it given a clue within 30 seconds.


\subsection{Data entry}

\subsection{Data maintenance}

\subsection{Procedure}

\subsection{Retrieval}



% Non-Functional Requirements
\newpage
\section{Non-Functional Requirements}


\subsection{Performance}

\subsection{Security}

\subsection{Legal and Access}

\subsection{Backup and Recovery}

\subsection{Archiving and Retention}

\subsection{Maintainability}

\subsection{Capacity}



% User Requirements
\newpage
\input{requirements/user_requirements}

% Assumptions 
\newpage
\section{Project Constraints}

\subsection{Mandated Constraints}

\subsection{Naming Conventions and Terminology}

\subsection{Relevant Facts and Assumptions}

% Risk Assessment
\newpage
\section{Risk Assessment}

% Introduction.
An investigation into the potential risks that may present themselves during the course of this project is of paramount importance if their impact is going to be minimised, should they become a reality.

% Included here a risk impact table
\subsection{Risk Identification}

% PI Scores and risk classification graph
\subsection{Risk Classification}

\subsection{Risk Prevention}

\subsection{Risk Mitigation}
