\chapter{Software Specification}
\label{chap:requirements}
\citet{cadle10} states that there is a standard hierarchical approach in 
structuring requirements. Table \ref{table:requirementsCategories} outlines
the main four categories:

\begin{table}[H]
  \begin{tabular}{|l|l|l|l|}
    \hline
    {\bf General} & {\bf Technical} & {\bf Functional} & {\bf Non-Functional} \\ 
    \hline
    Business constraints & Hardware & Data entry & Performance \\ 
    Business policies & Software & Data maintenance & Security \\ 
    Legal & Interoperability & Procedure & Legal and Access \\ 
    Branding & Internet & Retrieval & Backup and Recovery \\ 
    Cultural & ~ & ~ & Archiving and Retention \\ 
    Language & ~ & ~ & Maintainability \\ 
    ~ & ~ & ~ & Business Continuity \\ 
    ~ & ~ & ~ & Availability \\ 
    ~ & ~ & ~ & Usability \\ 
    ~ & ~ & ~ & Capacity \\
    \hline
  \end{tabular}
  \label{table:requirementsCategories}
\end{table}

Each of the above categories will form the basis of this requirements chapter,
along with any additiona user requiremens, assmptions and a risk assessment.

% General Requirements
\newpage
\section{General Requirements}

\subsection{Business constraints}

\subsection{Business policies}

\subsection{Legal}

\subsection{Branding}

\subsection{Cultural}

\subsection{Language}



% Technical Requirements
\newpage
\section{Technical Requirements}

\subsection{Hardware}

\subsection{Software}

\subsection{Interoperability}

\subsection{Internet}



% Functional Requirements
\newpage
\section{Functional Requirements}


\subsection{Data entry}

\subsection{Data maintenance}

\subsection{Procedure}

\subsection{Retrieval}



% Non-Functional Requirements
\newpage
\section{Non-Functional Requirements}

These are qualities the product must have.
e.g the product shall provide a please user experience.


\subsection{Performance}

\subsection{Security}

\subsection{Legal and Access}

\subsection{Backup and Recovery}

\subsection{Archiving and Retention}

\subsection{Maintainability}

\subsection{Capacity}



% User Requirements
\newpage
\section{User Requirements}


% Assumptions 
\newpage
\section{Assumptions}


% Risk Assessment
\newpage
\section{Risk Assessment}

% Introduction.
An investigation into the potential risks that may present themselves during the course of this project is of paramount importance if their impact is going to be minimised, should they become a reality.

% Included here a risk impact table
\subsection{Risk Identification}

% PI Scores and risk classification graph
\subsection{Risk Classification}

\subsection{Risk Prevention}

\subsection{Risk Mitigation}
