\subsection{Cryptic Crosswords} 

Most cryptic clues consist of two different parts, the word play and the
definition itself. The definition is like a clue found within a quick crossword
and the word play is an indication to the answer.

Clue types such as double definition and purely cryptic break the usual format
of cryptic clues. Double definitions miss out the word play whereas the purely
cryptic clues miss out the usual simpler definition and become a fully cryptic
definition. Other types of clue add to the usual format when smaller clues are
embedded within the larger clue to assist with the word play.

Punctuation within clues should always be disregarded unless it is a question
mark or an exclamation mark. Punctuation such as commas and hyphens are used to
distract the solver from the answer usually by attempting to dictate how the
clue is read.

A question mark tells the solver that the clue requires creative thinking to
work out the answer which could be witty in nature. An exclamation mark can mean
that the word play and the definition may intersect which is otherwise known as
a clue of the type ``\& lit''. Articles within clues can also be very important
and should not be disregarded when reading the clue.

Cryptic clues which have a particular tense will always be for a clue with the
same tense. Similarly a plural clue determines that the answer will also be
plural.
