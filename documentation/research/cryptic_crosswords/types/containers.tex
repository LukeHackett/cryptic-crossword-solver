\subsubsection{Containers}

A container clue includes three definitions and an indicator. One of the
definitions is for the final solution whereas the other definitions describe two
separate answers where one is contained within the other.

They are similar to charade clues in the way that other types of clues can be
used within container clues such as anagrams and charades themselves. Possible
indicators that denote a container clue are:

\begin{itemize} 
    \item Word indicators which could be used to indicate the inner word (e.g. 
    inside, held) or the outer word (e.g. outside, external)
\end{itemize}

\paragraph{Example:} \emph{Building for the workers in principle (8)} - \citep{shuchiContainers09} \\
\textbf{Answer:} TENEMENT 

\begin{itemize}
    \item ``principle'' is the definition for the outer word which gives tenet 
    \item ``in'' is the indicator that the answer for the inner word will be 
    placed within the outer word (tenet) 
    \item ``the workers'' is the definition for the inner word which is men 
    \item `Building' is the overall definition which could mean tenement which 
    is also given when men is put within tenet 
\end{itemize}