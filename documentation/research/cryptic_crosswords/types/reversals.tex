\subsubsection{Reversals}

A reversal requires the solver to reverse a number of letters to give a new
word. The clue consists of a definition, an indicator that the clue is a
reversal clue and some ``fodder'' which is a phrase or word which could contain
the letters to be reversed or a smaller clue which leads to the letters which
need to be reversed. Possible indicators that denote a reversal clue are:

\begin{itemize}
    \item Words/phrases used for directional purposes e.g. left, up 
    \item The word indicators may also be relative to the direction the clue 
    should be placed within the crossword (down or across) 
\end{itemize}

\paragraph{Example:} \emph{Stop the flow in crazy get-up (3)} - \citep{shuchiReversals08} \\
\textbf{Answer:} DAM 

\begin{itemize}
    \item ``Stop the flow'' is the definition 
    \item ``get-up'' is the indicator that the clue is a reversal 
    \item Another word for ``crazy'' is mad which reversed gives the answer dam 
\end{itemize}