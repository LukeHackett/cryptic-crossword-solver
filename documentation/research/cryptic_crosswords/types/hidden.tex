\subsubsection{Hidden}  

The answer for a hidden clue is concealed within the clue itself and can be
spread over more than one word as well as possibly being hidden in reverse. The
clue will have a definition, an indicator that the answer is hidden within the
clue and a word or set of words which have the answer in them. Possible 
indicators that denote a hidden clue are:

\begin{itemize} 
    \item Word/s e.g. contains, in, within, held by, from 
    \item Large words with a hidden word indicator before it may have the 
    answer inside them 
    \item A clue which seems inelegantly written or a clue which contains 
    proper nouns  
\end{itemize}

\paragraph{Example 1:} \emph{Metal concealed by environmentalist (4)} - \citep{shuchiHidden08} \\
\textbf{Answer 1:} IRON 

\begin{itemize}
    \item ``Concealed by'' is a phrase indicator for hidden clues 
    \item ``Environmentalist'' is a large word with an indicator in front 
    \item ``Metal'' is the definition so the answer is a type of metal which 
    can be found within the word ``environmentalist'', hence iron 
\end{itemize}


\paragraph{Example 2:} \emph{Mountain range in central Taiwan (5)} - \citep{shuchiHidden08} \\
\textbf{Answer 2:} ALTAI 

\begin{itemize}
    \item ``in'' is a word indicator for hidden clues 
    \item ``central Taiwan'' contains a proper noun which indicates the answer 
    is hidden here 
    \item ``Mountain range'' is the definition 
    \item Without knowledge of mountain ranges the answer could be narrowed 
    down to the following words (assuming ``central'' would not be within the 
    clue without a purpose):
    \begin{itemize}
        \item TRALT 
        \item RALTA 
        \item ALTAI 
        \item LTAIW
    \end{itemize} 
    \item If some of the crossword is completed within the area this clue is 
    placed, the correct answer could be found through trial and error.
\end{itemize}
