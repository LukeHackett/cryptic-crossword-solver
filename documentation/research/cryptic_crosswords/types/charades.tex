\subsubsection{Charades}

A charade clue forms its answer with the use of smaller answers to smaller clues
within the main clue. Abbreviations and first/last letters of words are common
within charade clues to make up the complete answer.

Two or three parts are usually within the charade clue to solve the correct
answer, they may not be in the right order however, and word indicators will be
used to warn the solver. 

Other types of clues can also be used within charade clues for the different
parts such as reversals and homophones, if this is the case there will be
indicators for the specific type. Possible  indicators that denote a charade
clue are:

\begin{itemize}
    \item Words e.g. with, follows, behind, after to indicate joining of 
    answers to parts of the clue 
\end{itemize}

\paragraph{Example:} \emph{Prior belted one that is ultimately right (7)} - \citep{shuchiCharades08} \\
\textbf{Answer:} EARLIER

\begin{itemize}
    \item ``Prior'' is the definition 
    \item ``belted one'' gives earl 
    \item ``that is'' gives the abbreviation for i.e. or ie 
    \item ``right'' gives the abbreviation for r 
    \item All the segments put together give the word ``earlier'' which can 
    also mean ``prior''
\end{itemize}
