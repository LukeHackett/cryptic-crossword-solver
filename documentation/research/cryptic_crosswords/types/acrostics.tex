\subsubsection{Acrostics}

An acrostic clue commonly involves picking the first letter from a group of
words and putting them together to form the answer. It is possible that the clue
will require the last or middle letters from words to solve them or that the
letters should be put together in reverse order.

This clue type has a definition and an indicator as well as ``fodder'' which in
this case means the group of words the necessary letters will come from.
Possible indicators that denote an acrostic cryptic  clue are:

\begin{itemize}
    \item Words which could mean start or beginning 
    \item If the clue is unusually long and so is the number indicator to 
    determine how long the answer should be 
\end{itemize}

\paragraph{Example:} \emph{Some URLs recommended for beginners to explore online (4)} - \citep{shuchiAcrostics08} \\
\textbf{Answer:} SURF 

\begin{itemize}
    \item ``beginners'' is the indicator 
    \item As the clue states the answer should be of length four it is assumed 
    ``to explore online'' is the definition as there are only three possible 
    letters for an acrostic in the phrase 
    \item ``Some URLs recommended for'' is the ``fodder'' for the clue. The 
    indicator implies that the first letters should be taken from the first 
    letters of the words within the ``fodder'', hence surf  
\end{itemize}