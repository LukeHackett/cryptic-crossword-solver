\section{Crosswords}

Arthur Wynne produced the first crossword puzzle which was printed on December
21st 1913. A crossword is a puzzle which involves the solver resolving the
answer to a clue and placing it in the correct space within the grid. 

A grid is made up of black and white squares, the black squares are blanks and 
the white squares are where the solver must place the answers. A crossword grid 
comes with a set of clues. The clues are usually arranged based upon their 
positing within the grid. Clues that that appear downwards in the grid are kept 
separate to the clues that appear across the grid. 

The white squares which are used for the first letter of an answer to a clue 
usually have a number in the top left hand corner to indicate the clue which 
links to this area of the grid.

There are different types of crosswords such as quick, cryptic and double-clue.
A quick crossword has clues which simply define the answer. A cryptic crossword
is more complex as it has word play as well as simple definitions and many
different types of clues. A double-clue crossword combines the two and allows
for a simpler option for the solver when the cryptic clues become too difficult.


% Crossword Users
\newpage
\subsection{Users}

Kathryn Friedlander and Philip Fine \citep{friedlander09} carried out an
investigation into whether the amount of cryptic crosswords completed by a
solver determined how successful they were at them. 

To complete this study they gathered data from 241 people, this data can be 
used as part of research to determine the typical audience cryptic crosswords 
have. 

The following facts about the user base are taken from the results section of 
the paper \citep{friedlander09}:

\begin{itemize}
    \item ``209 M, 32 F''
    \item ``mean age=53 years, range=23-83''
    \item ``mean time spent=8 hours per week, range=1-30''
\end{itemize}


\subsection{Cryptic Crosswords}
Most cryptic clues consist of two different parts, the word play and the definition itself. The definition is like a clue found within a quick crossword and the word play is an indication to the answer. Clue types such as double definition and purely cryptic break the usual format of cryptic clues. Double definitions miss out the word play whereas the purely cryptic clues miss out the usual simpler definition and become a fully cryptic definition. Other types of clue add to the usual format when smaller clues are embedded within the larger clue to assist with the word play.

Punctuation within clues should always be disregarded unless it is a question mark or an exclamation mark. Punctuation such as commas and hyphens are used to distract the solver from the answer usually by attempting to dictate how the clue is read. A question mark tells the solver that the clue requires creative thinking to work out the answer which could be witty in nature. An exclamation mark can mean that the word play and the definition may intersect which is otherwise known as a clue of the type \lq\lq \& lit\rq\rq. Articles within clues can also be very important and should not be disregarded when reading the clue.

Cryptic clues which have a particular tense will always be for a clue with the same tense. Similarly a plural clue determines that the answer will also be plural.

\subsection{Crossword Clue Types}
An important skill needed to solve a cryptic crossword is to be able to spot the type of clue given. Below is a list of the most common types of cryptic clue and expected rules the clue should follow so they are identifiable. 

\subsubsection{Purely Cryptic}
Although clues within a cryptic crossword usually include both a definition and word play, this type of clue is an exception because the whole clue is a definition written in an unusual way. Word play within other clue types assist the user in being able to determine their answer is correct as well as solve them, because there is no word play more than one answer could be found which are incorrect.  

Below are possible indicators for this type of clue:
\begin{itemize} 
	\item Question mark 
	\item Exclamation mark
\\
\end{itemize} 

Clue Example: \emph{Frames for summer's activities? (5)} - \citep{shuchiCryptic08}

Answer: ABACI 

\begin{itemize}
	\item \lq\lq Summer\rq\rq as in a person who does mathematical sums 
	\item An abacus is a frame which holds moving beads 
	\item As the clue is plural so must the answer be, hence abaci
\end{itemize}

\subsubsection{Hidden} 
The answer for a hidden clue is concealed within the clue itself and can be spread over more than one word as well as possibly being hidden in reverse. The clue will have a definition, an indicator that the answer is hidden within the clue and a word or set of words which have the answer in them.  

Below are possible indicators for this type of clue:
\begin{itemize} 
	\item Word/s e.g. contains, in, within, held by, from 
	\item Large words with a hidden word indicator before it may have the answer inside them 
	\item A clue which seems inelegantly written or a clue which contains proper nouns  
\\
\end{itemize}

Clue Example: \emph{Metal concealed by environmentalist (4)} - \citep{shuchiHidden08}

Answer: IRON 

\begin{itemize}
	\item \lq\lq Concealed by\rq\rq is a phrase indicator for hidden clues 
	\item \lq\lq Environmentalist\rq\rq is a large word with an indicator in front 
	\item \lq\lq Metal\rq\rq is the definition so the answer is a type of metal which can be found within the word \lq\lq environmentalist\rq\rq, hence iron 
\\
\end{itemize}

Clue Example: \emph{Mountain range in central Taiwan (5)} - \citep{shuchiHidden08}

Answer: ALTAI 

\begin{itemize}
	\item \lq\lq in\rq\rq is a word indicator for hidden clues 
	\item \lq\lq central Taiwan\rq\rq contains a proper noun which indicates the answer is hidden here 
 	\item \lq\lq Mountain range\rq\rq is the definition 
 	\item Without knowledge of mountain ranges the answer could be narrowed down to the following words (assuming \lq\lq central\rq\rq would not be within the clue without a purpose): 
	\begin{itemize}
		\item TRALT 
		\item RALTA 
		\item ALTAI 
		\item LTAIW
	\end{itemize} 
\item If some of the crossword is completed within the area this clue is placed, the correct answer could be found through trial and error 
\end{itemize}

\subsubsection{Charades}

A charade clue forms its answer with the use of smaller answers to smaller clues within the main clue. Abbreviations and first/last letters of words are common within charade clues to make up the complete answer. Two or three parts are usually within the charade clue to solve the correct answer, they may not be in the right order however, and word indicators will be used to warn the solver. Other types of clues can also be used within charade clues for the different parts such as reversals and homophones, if this is the case there will be indicators for the specific type.  

Below are possible indicators for this type of clue: 
\begin{itemize}
	\item Words e.g. with, follows, behind, after to indicate joining of answers to parts of the clue 
\\
\end{itemize}

Clue Example: \emph{Prior belted one that is ultimately right (7)} - \citep{shuchiCharades08}

Answer: EARLIER 

\begin{itemize}
	\item \lq\lq Prior\rq\rq is the definition 
	\item \lq\lq belted one\rq\rq gives earl 
	\item \lq\lq that is\rq\rq gives the abbreviation for i.e. or ie 
	\item \lq\lq right\rq\rq gives the abbreviation for r 
	\item All the segments put together give the word \lq\lq earlier\rq\rq which can also mean \lq\lq prior\rq\rq
\end{itemize}

\subsubsection{Anagrams}

Anagram clue types have a definition, a word or phrase to indicate the clue is of this type and an element called \lq\lq fodder\rq\rq. An anagram is a word whose letters can be rearranged to form another word; within an anagram clue the letters to rearrange are known as \lq\lq fodder\rq\rq and are placed next to the indicator.  

Below are possible indicators for this type of clue:
\begin{itemize} 
	\item Words which could mean change or shifting 
	\item A clue which seems inelegantly written or a clue which contains proper nouns
\\
\end{itemize}

Clue Example: \emph{Toy breeds trained to find out a place for pearls (6,3)} - \citep{shuchiAnagram08} 

Answer: OYSTER BED 

\begin{itemize}
	\item \lq\lq trained\rq\rq indicates the clue is an anagram   
	\item \lq\lq Toy breeds\rq\rq is an abnormal phrase and is the \lq\lq fodder\rq\rq of the clue 
	\item \lq\lq to find out a place for pearls\rq\rq is left to become the definition 
	\item \lq\lq Toy breeds\rq\rq is then moved around to give oyster bed 
\end{itemize}

\subsubsection{Homophones}

A homophone is a word which sounds like another word but has a separate meaning. This type of clue has a definition, a word or phrase which means the same as the homophone to find and an indicator. 

Below are possible indicators for this type of clue:
\begin{itemize} 
	\item Words which indicate hearing or sound e.g. said, heard 
	\item Normally the indicator for a homophone is next to the word or phrase which is to be used to find a homophone  
\\
\end{itemize}

Clue Example: \emph{Refer to a location, reportedly (4)} - \citep{shuchiHomophone08}

Answer: CITE 

\begin{itemize}
	\item \lq\lq reportedly\rq\rq is the homophone indicator 
	\item \lq\lq a location\rq\rq is the phrase which needs to be used to find a homophone 
	\item\lq\lq Refer to\rq\rq is the definition 
	A location can be otherwise known as a \lq\lq site\rq\rq, hence cite 
\end{itemize}

\subsubsection{Acrostics}

An acrostic clue commonly involves picking the first letter from a group of words and putting them together to form the answer. It is possible that the clue will require the last or middle letters from words to solve them or that the letters should be put together in reverse order. This clue type has a definition and an indicator as well as \lq\lq fodder\rq\rq which in this case means the group of words the necessary letters will come from. 

Below are possible indicators for this type of clue:
\begin{itemize}
 	\item Words which could mean start or beginning 
	\item If the clue is unusually long and so is the number indicator to determine how long the answer should be 
\\
\end{itemize}

Clue Example: \emph{Some URLs recommended for beginners to explore online (4)} - \citep{shuchiAcrostics08}

Answer: SURF 

\begin{itemize}
	\item \lq\lq beginners\rq\rq is the indicator 
	\item As the clue states the answer should be of length four it is assumed \lq\lq to explore online\rq\rq is the definition as there are only three possible letters for an acrostic in the phrase 
	\item \lq\lq Some URLs recommended for\rq\rq is the \lq\lq fodder\rq\rq for the clue. The indicator implies that the first letters should be taken from the first letters of the words within the \lq\lq fodder\rq\rq, hence surf  
\end{itemize}
 
\subsubsection{Palindromes}

A palindrome is a word which reads and looks the same when it is reversed. This type of clue has an indicator and a definition. 

Below are possible indicators for this type of clue:
\begin{itemize} 
	\item Phrases which may mean either way or going around in circles
\\
\end{itemize}

Clue Example: \emph{Unacceptable, going up or down (3,2)} - \citep{connorPalindromes12} 

Answer: NOT ON 

\begin{itemize}
	\item \lq\lq going up or down\rq\rq is an indicator as it could mean \lq\lq in either direction\rq\rq like the format of a palindrome 
	\item \lq\lq Unacceptable\rq\rq is then left as the definition which gives the answer, not on 
\end{itemize}

\subsubsection{Reversals}

A reversal requires the solver to reverse a number of letters to give a new word. The clue consists of a definition, an indicator that the clue is a reversal clue and some \lq\lq fodder\rq\rq which is a phrase or word which could contain the letters to be reversed or a smaller clue which leads to the letters which need to be reversed.  

Below are possible indicators for this type of clue: 
\begin{itemize}
	\item Words/phrases used for directional purposes e.g. left, up 
	\item The word indicators may also be relative to the direction the clue should be placed within the crossword (down or across) 
\\
\end{itemize}

Clue Example: \emph{Stop the flow in crazy get-up (3)} - \citep{shuchiReversals08}

Answer: DAM 

\begin{itemize}
	\item \lq\lq Stop the flow\rq\rq is the definition 
	\item \lq\lq get-up\rq\rq is the indicator that the clue is a reversal 
	\item Another word for \lq\lq crazy\rq\rq is mad which reversed gives the answer dam 
\end{itemize}

\subsubsection{\lq\lq \& lit\rq\rq}

\lq\lq \& lit\rq\rq clues, which means \lq\lq and literally so\rq\rq, is a type of clue where the definition and the word play are the same and are not split out into separate phrases or words as with other clues. The definition is the whole clue and the word play can be one or more of any of the normal clue types such as anagrams and charades. 

Below are possible indicators for this type of clue:
\begin{itemize} 
	\item Exclamation mark
\\
\end{itemize}

Clue Example: \emph{Cop in male form (9)}   - \citep{shuchiLit08} 

Answer: POLICEMAN 

\begin{itemize}
	\item The whole clue is the definition 
	\item \lq\lq form\rq\rq is an indicator for an anagram clue 
	\item \lq\lq Cop in male\rq\rq can be rearranged to policeman which is also the answer to a \lq\lq Cop in male form\rq\rq 
\end{itemize}

\subsubsection{Double Definition}

A double definition clue has no word play and is purely a clue with two (or possibly more) definitions which lead to the same answer. 

Below are possible indicators for this type of clue:
\begin{itemize} 
	\item Possibly shorter than most clues (2 or 3 words) 
	\item Although it is advisable to ignore punctuation when solving cryptic clues, a double definition may have a piece of punctuation separating the definitions 
\\
\end{itemize}

Clue Example: \emph{Robust author (5)}  - \citep{shuchiDouble08}

Answer: HARDY 

\begin{itemize}
	\item Both words are separate definitions which could both mean hardy 
\end{itemize}

\subsubsection{Containers}

A container clue includes three definitions and an indicator. One of the definitions is for the final solution whereas the other definitions describe two separate answers where one is contained within the other. They are similar to charade clues in the way that other types of clues can be used within container clues such as anagrams and charades themselves.  

Below are possible indicators for this type of clue:
\begin{itemize} 
	\item Word indicators which could be used to indicate the inner word (e.g. inside, held) or the outer word (e.g. outside, external)
\\
\end{itemize}

Clue Example: \emph{Building for the workers in principle (8)}  - \citep{shuchiContainers09}

Answer: TENEMENT 

\begin{itemize}
	\item \lq\lq principle\rq\rq is the definition for the outer word which gives tenet 
	\item \lq\lq in\rq\rq is the indicator that the answer for the inner word will be placed within the outer word (tenet) 
	\item \lq\lq the workers\rq\rq is the definition for the inner word which is men 
	\item \lq\lq Building\rq\rq is the overall definition which could mean tenement which is also given when men is put within tenet 
\end{itemize}

\subsubsection{Deletions}

Deletion clues require the solver to retrieve the answer by looking at the definition and the word play and removing the correct letters from the correct word. The clue will not usually have the word which needs letters removing from it directly within the clue.  

Below are possible indicators for this type of clue:
\begin{itemize}
 	\item Words to indicate letters should be removed from a certain place within the word such as its first or last letter 
	\item Words to indicate certain letters should be removed from the word found. These could be words that can be abbreviated 
\\
\end{itemize}

Clue Example: \emph{Little shark edges away from diver's equipment (3)}  - \citep{shuchiDeletions09}  

Answer: CUB 

\begin{itemize}
	\item \lq\lq edges away from\rq\rq indicates that the first and the last letter should be removed from the answer which comes from \lq\lq diver’s equipment\rq\rq
	\item \lq\lq diver’s equipment\rq\rq is otherwise known as scuba  
	\item Removing the \lq\lq s\rq\rq and the \lq\lq a\rq\rq from scuba leaves the word cub which is also the answer given from the definition \lq\lq Little shark\rq\rq
\end{itemize}

\subsubsection{Spoonerisms}

A spoonerism clue has a definition and word play which is usually a phrase which describes another. This phrase, which is usually two words long, is taken and the first letter of each is swapped around to gain a new phrase which is in turn the answer to the clue.  

Below are possible indicators for this type of clue:
\begin{itemize} 
	\item The word \lq\lq Spooner\rq\rq or \lq\lq Spooner’s\rq\rq is placed within the clue
\\ 
\end{itemize}

Clue Example: \emph{Spooner's cheerful enthusiast? He'll get you across (8)} - \citep{connorSpoon12}

Answer: FERRYMAN 

\begin{itemize}
	\item \lq\lq Spooner’s\rq\rq indicates that this clue is a spoonerism 
	\item \lq\lq He’ll get you across\rq\rq is used as the definition 
	\item \lq\lq cheerful enthusiast\rq\rq could also be known as a merry fan 
	\item Swapping the first letters from the words merry and fan give the answer ferryman 
\end{itemize}

\subsubsection{Pattern}

This type of clue has a definition, an indicator that the clue is of the type pattern and a phrase or word which has the answer within them arranged as a pattern. These patterns can be odd or even letters joined together to make a word or possibly letters picked from regular intervals. Pattern word play can also be joined together with other types of clues such as charades to form a more complex answer. 

Below are possible indicators for this type of clue:
\begin{itemize} 
	\item Words which could mean even, odd or routine 
\\
\end{itemize}

Clue Example: \emph{Beasts, free, ginned, we hear — regular losses there! (8)} - \citep{shuchiPicking09}

Answer: REINDEER 

\begin{itemize}
	\item \lq\lq regular\rq\rq indicates it is a pattern clue as well as \lq\lq losses\rq\rq to indicate the dropping of letters 
	\item \lq\lq Beasts\rq\rq is the definition 
	\item \lq\lq free, ginned, we hear\rq\rq holds the answer reindeer by picking out the first \lq\lq r\rq\rq within free and each letter alternately from then on
\end{itemize} 

\subsubsection{Substitutions}

A substitution clue involves removing letters from a word and replacing it with another to retrieve the answer. There are two definitions within the clue, one definition to retrieve the word to substitute letters from and another to define the final answer. The letter or letters to substitute are usually an abbreviation which can be found within the clue itself. 

Below are possible indicators for this type of clue:
\begin{itemize} 
	\item Words which mean substitution e.g. replace, switch, exchange
\\
\end{itemize}

Clue Example: \emph{Unexciting story gets mark for length (4)} - \citep{shuchiSubstitutions08}

Answer: TAME 

\begin{itemize}
	\item \lq\lq mark for length\rq\rq is an indicator for a substitution clue  
	\item \lq\lq mark\rq\rq can be abbreviated to \lq\lq m\rq\rq and \lq\lq length\rq\rq can be abbreviated to \lq\lq l\rq\rq, therefore replace \lq\lq l\rq\rq with \lq\lq m\rq\rq 
	\item \lq\lq story\rq\rq is the definition for the word which needs the substitution and could be defined as a tale 
	\item Replacing the \lq\lq l\rq\rq in tale with \lq\lq m\rq\rq gives the word tame which can also mean unexciting 
\end{itemize}

\subsubsection{Shifting}

A shifting clue has an indicator, a definition of the final answer and another definition for the word which needs to be used to shift a letter to a different position within the word to find the final answer. The shifting of a letter could be moving the first letter to the last position in the word or in a more complex clue letters could be shifted within the middle of the word.  

Below are possible indicators for this type of clue:
\begin{itemize} 
	\item Words e.g. shift, change, move 
	\item Phrases e.g. head to foot 
\\
\end{itemize}

Clue Example: \emph{Character needs help, head to foot (4)}  - \citep{shuchiShifting09} 

Answer: BETA 

\begin{itemize}
	\item \lq\lq head to foot\rq\rq indicates moving a letter from the front of a word to the end 
	\item \lq\lq help\rq\rq is the definition for the word which requires letter shifting and can also be defined as abet 
	\item Moving the first letter of \lq\lq abet\rq\rq to the end gives the word beta which is a \lq\lq Character\rq\rq
\end{itemize}

\subsubsection{Exchange}

An exchange clue is similar to a shifting clue however instead of only one letter shifting positions within a word, two letters within a word exchange places to form a new word. Typically the letters to exchange will be the first and last letters of a word or two letters next to each other, however it is possible more than one letter on each side will need to be swapped. For example, the word \lq\lq rage\rq\rq can be split into two sections \lq\lq ra\rq\rq and \lq\lq ge\rq\rq which can then be exchanged to make the word \lq\lq gear\rq\rq.   

Below are possible indicators for this type of clue:
\begin{itemize} 
	\item Words e.g. swap, exchange, change 
\\
\end{itemize}



Clue Example: \emph{Doomed king switching sides? True (4)}  - \citep{shuchiExchange09}

Answer: REAL 

\begin{itemize}
	\item \lq\lq switching sides\rq\rq indicates that this clue is an exchange clue 
	\item A \lq\lq Doomed king\rq\rq can also be known as a \lq\lq Lear\rq\rq
	\item \lq\lq True\rq\rq is the definition which can also be defined as \lq\lq real\rq\rq which can be gained by exchanging the first letter of \lq\lq Lear\rq\rq (\lq\lq L\rq\rq) with the last letter (\lq\lq r\rq\rq) 
\end{itemize}