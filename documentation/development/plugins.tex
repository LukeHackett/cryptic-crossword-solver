%%%
%% Development :: Plug-ins
%%%
\section{Plug-ins}
\label{sec:plugins}

Section \ref{sec:plug_and_play_architecture} on page
\pageref{sec:plug_and_play_architecture} described the overall system's plug and
play architecture and how it has helped the overall project development phase.
Within this section an in depth description and small example will be presented
to help future developers create a new solver that can be plugged into the
system.

All solvers must be placed within the \texttt{uk.ac.hud.cryptic.solver} package.
This package will be the base package in which the run time class importer will
attempt to import the various solver classes.

Once developed, the new solver class must be added (with it's full package name)
to the properties file which was described in section 
\ref{sec:plug_and_play_architecture}.

Each solver must extend the \texttt{Solver} base abstract class. The newly 
extended class can have various methods but must implement the \texttt{solve} 
and \texttt{toString} methods.

The \texttt{toString} method should return a String representation of the type 
of solver that the solver is, for example ``container'' or ``anagram''. The 
\texttt{solve} method will be automatically called by the system, and the core 
algorithm must be placed within this method (or utilise class specific methods).

Listing \ref{simpleSolver} illustrates a basic acrostic solver template.

\begin{lstlisting}[caption={An example user-defined solver}, label=simpleSolver]  
public class MyAcrosticSolver extends Solver {

  public MyAcrosticSolver() {
    super();
  }

  public MyAcrosticSolver(Clue clue) {
    super(clue);
  }

  @Override
  public SolutionCollection solve(Clue c) {
    // Add the algorithm here
  }

  @Override
  public String toString() {
    return "acrostic";
  }

}
\end{lstlisting}

As shown in listing \ref{simpleSolver} the solve method returns a 
SolutionCollection which is added to an overall collection to deduce the likely
correct solution to the clue. This was described within sub-section 
\ref{sub:manager} on page \pageref{sub:manager}.
