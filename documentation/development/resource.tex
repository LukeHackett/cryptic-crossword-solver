%%%
%% Development :: Resources
%%%
\section{Resources}
\label{sec:resources}

There are a number of resources which were needed to aid with solving 
the various clue types. As a lot of the solvers shared most of the resources 
a design decision was made to keep the resources in their own package for 
all solvers to access them as and when they need to.

%%%%
%% Development :: Resources :: Abbreviations
%%%
\subsection{Abbreviations}

Some clue types, for example the `Charade' clue type, require the knowledge 
of the different abbreviations that come with certain words. To provide this 
knowledge a file was found with a list of words and abbreviations in JSON 
format (JavaScript Object Notation). 

Listing \ref{abbr} illustrates the way in which the abbreviations for `quiet' 
are displayed in the file:

\begin{lstlisting}[caption={A sample of the abbreviations file},
                   label=abbr]  
 "quiet": [
  "p", 
  "pp", 
  "sh", 
  "mum"
 ], 
\end{lstlisting}

The Abbreviations class reads in all the possible abbreivations from the file 
at run time and stores them within a map which includes the word to get the 
abbreviations for and it's abbreviations as a set. 

There are two ways in which abbreviations could be needed to solve a clue, 
one way is to retrieve abbreivations for one word and the other is to 
retrieve abbreviations for a phrase or a whole clue. 

The method to retrieve abbreviations for one single word simply returns the 
set of abbreviations for a given word (or key) in the map if it exists within the 
file. 

The method to get abbreviations for a phrase or a whole clue gets the 
abbreviations for as many words as possible in the given clue.
For example, "help the medic" will contain seven abbreviations for the word
 medic. "medal for the medic" will contain four abbreviations for medal and seven
 for medic. However, the clue "master of ceremonies" will return one
 abbreviation which matches the entire clue (i.e. "master of ceremonies").
 The algorithm is greedy, and will attempt to match the biggest String
possible in the given clue. This means it will match all of the String
 "master of ceremonies" before matching abbreviations for "master".

%%%%
%% Development :: Resources :: Dictionary
%%%
\subsection{Dictionary}

The dictionary is an essential resource for the solving of clues to 
programmatically determine whether a string of letters is a valid word. 
A text file was found with a list of words within it which is read into the 
Dictionary class when an instance is created.

Listing \ref{dict} illustrates simply how the words in the dictionary file are
displayed:

\begin{lstlisting}[caption={A sample of the dictionary file},
                   label=dict]  
abaci
aback
abacs
abactinal
abactinally
abactor
abactors
abacus
\end{lstlisting}

Once the file has been read in, there are a custom list of words to 
exclude from the dictionary and a list of words that need to be added 
found from solving particular clues. 

One of the methods used regularly is the filtering method which removes any
 words from a given collection that are not present in the dictionary. This is an
 effective way to remove words that have being constructed by a solver algorithm
 which are essentially just an assortment of letters which hold no identified meaning.
 A similar method is used to filter any prefixes passed in within a collection that are 
also not held in the dictionary.

When potential solutions have been found by a solver, it is necessary to ensure 
the algorithm has returned solutions that fit the end solutions pattern. Requirements 
such as the length input by the user and any known letters input by the user must 
be adhered to. In the dictionary class there is a method which gets all word matches 
within the dictionary for a given solution pattern. As with filtering solutions, there is 
also a method to match up all words in the dictionary that have the same prefix as 
a prefix passed in.

There are also simple methods solvers can used to identify whether a single word or a
 phrase is contained within the dictionary simply by checking whether the collection the 
dictionary file has been read into contains the word or not. 

%%%%
%% Development :: Resources :: Thesaurus
%%%
\subsection{Thesaurus}

For clue types which do not have the answer itself nested within it, it is usually necessary 
to take the clue words and find synonyms for them. This is where the necessity for a thesaurus 
applies to aid the algorithms in solving clues. A thesaurus file was found which holds a vast number 
of entries where each word after the first word in an entry is a synonym of the first word. 

Listing \ref{thes} illustrates how the words in the thesaurus file are
displayed with the example word `dank':

\begin{lstlisting}[caption={A sample of the thesaurus file for the word `dank'},
                   label=thes]  
dank,boggy,damp,dampish,dewy,fenny,humid,marshy,moist,muggy,rainy,roric,roriferous,sticky,swampy,tacky,undried,wet,wettish
\end{lstlisting}

As with the other resources the file is read in and stored within a collection, this 
time in the form of a map where the first word is stored within an entry along with it's 
synonyms. 

There are a wide range of different methods that can be used to retrieve a different array 
of synonyms from vague to specific. Below is a list of functionality that has been written 
to retrieve synoynms:

  \begin{itemize}
    \item Obtain a list of "synonyms of a word's synonyms" to increase the chances of
 finding the correct solution. These must match against a supplied pattern. 
    \item Obtain a list of "synonyms of a word's synonyms" to increase the chances of
 finding the correct solution. These must match a minimum and maximum length passed 
in for the synonym.
    \item Retrieve all single word synonyms of a given word with a maximum and minimum length 
(because some synonyms can be more than one word long) 
    \item Retrieve all synonyms of a given word with no pattern or minimum or maximum length
    \item Get the synonyms for as many words as possible in the given clue. For example,
 in the clue "help the medic", look for synonyms of "help the medic", "help the", "the medic", "help", "the", "medic".
    \item Retrieve all synonyms in the same entry in the thesaurus as a given word. This means to not 
 only look at the first word of an entry for synonyms, look through all entries and return every entry which
 contains a given word.
    \item Retrieve all synonyms in the same entry in the thesaurus as a given word which match against the given pattern.
    \item Check if a given potential solution found by a solver matches as a synonym against
 any of the words present in the clue.
  \end{itemize}

%%%%
%% Development :: Resources :: Homonym Dictionary
%%%
\subsection{Homonym Dictionary}

The `Homophone' clue type requires a resource for retrieving homonyms for words. 
A file was found which lists words with a string representation of the words pronunciation 
and as with the other resources, the file is read in and stored within a collection holding 
the word and an list of the pronunciation split into chunks. However, an additional collection 
is formed which is essentially a reverse of the first collection created. This allows pronunciations
 to be looked up faster. For example, looking up the pronunciation "HH AH0 L OW1" will return
 "hello". This saves having to iterate the entire homophone map to search for words with the
same pronunciation.

Listing \ref{homonym} illustrates how the words in the homonym dictionary file are
displayed with the example word `hello':

\begin{lstlisting}[caption={A sample of the homonym dictionary file for the word `hello'},
                   label=homonym]  
HELLO  HH AH0 L OW1
\end{lstlisting}

The class allows the algorithm to retrieve the pronunciation of a given word as well as 
get words which share the same pronunciation as the supplied word. This only works for words
 which share the exact same pronunciation.

%%%%
%% Development :: Resources :: Categoriser
%%%
\subsection{Categoriser}