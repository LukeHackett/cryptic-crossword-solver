%%%%
%% Design :: Primary Path
%%%
\section{Primary Path}
\label{sec:primary_path}

Before being able to correctly design a large, complex system a good 
understanding of the main scenario through through the system will be required.
The main scenario is ``a sequence of event or actions'' \citep{lunn03} in which
many smaller scenarios may occur.

The main scenario of the proposed cryptic crossword solver system is outlined 
below:

\begin{enumerate}
  \item Input the cryptic clue to be solved
  \item Input the pattern of the solution
  \item Process the data based upon a number of algorithms
  \item Output the most confident results
\end{enumerate}

Based upon the previously given main scenario, the primary path of the entire 
system can be proposed. The primary path is defined as the most commonly used 
`route' though a system with as few variances as possible \citep{lunn03}.

The proposed primary path for the cryptic crossword solver system is outlined 
below:

\begin{enumerate}
  \item Input the cryptic clue to be solved
  \item Input the pattern of the solution
  \item Determine the type of clue that has been given (e.g. anagram)
  \item Run the given clue type algorithm 
  \item Rank each of the results based upon a pre-defined criteria
  \item Output the top ranking results
\end{enumerate}

The primary path outlined above takes a relatively generic approach with regards
to how a clue should be solved. Ideally the clue would be categorised into a 
certain type of clue, thus requiring that particular solver to be run.

However the primary path does not illustrate the steps that are required to be 
taken if the clue can not be categorised --- does the system simply stop or 
would it try an alternative approach?

These additional steps may often be `optional', and hence are described as 
alternative paths. The alternative path will contain small and subtle changes 
to the primary path \citep{lunn03}.

If the system is unable to determine the type of clue that has been then an 
alternative approach is to be used. In this instance, the primary path would 
house the following additions and changes:

\begin{enumerate}
  \item[3] Determine the type of clue that has been given (e.g. anagram)
  \begin{enumerate}
    \item[3.1] Start a new instance of each solver
  \end{enumerate}
  \item[4] Run the each of the solvers separately
\end{enumerate}

The above alternative path shows that if the clue can not be categorised, then 
the system will attempt to solve the clue utilising a brute force approach. The 
above alternative path, could also be regarded as a `worst case scenario'.

The outlined primary and alternative paths are a simple way of illustrating the 
the general ideas behind the system. These paths will be actively referred to 
throughout the design process.
