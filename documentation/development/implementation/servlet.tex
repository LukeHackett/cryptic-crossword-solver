%%%%
%% Implementation :: Web Service & Servlets
%%%
\section{Web Service \& Servlets}
\label{sec:servlet}

The core system is wrapped around a RESTful web service, that allows users from
various devices to submit clues to be solved. Within this section both the web 
service and the servlet implementations will be discussed and presented.


%%%%
%% Implementation :: Web Service & Servlets :: Web Service
%%%
\subsection{Web Service}
\label{sub:servlet_web_service}

During the project analysis phase the decision was taken that the system's
functionality will be delivered via a web service. The web service was developed
using Java Enterprise Edition (Java EE) and Tomcat 7.

The main reason for using Java is that the web service could easily make use of
the various packages that are provided by the chosen natural language processing
library --- Apache OpenNLP written in Java.

The web service has solely been produced using the Java EE platform and does not
use any additional frameworks or libraries such as Apache Axis. The reason being
is that the Java EE platform will run on any machine that is capable of running
the Java virtual machine without any additional configuration.

Although Apache Axis (for example) provides additional functionality in
configuring the web service, it was decided that the project was to focus upon
the solving of a clue. Therefore a `standard web service' setup would easily
meet the requirements of the project.

Another design decision was taken to ensure that the web service followed a
RESTful style of communication. The mains reason for this is that some of the
target devices (i.e. mobile and tablet platforms) do not support SOAP based
communication without additional plug-ins. RESTful web services also provide a
number of advantages over their SOAP-based counter parts, as was highlighted
within the research section.


%%%%
%% Implementation :: Web Service & Servlets :: Servlets
%%%
\subsection{Servlets}
\label{sub:servlet_servlets}

The servlet design has been split across two classes --- \texttt{Servlet} and 
\texttt{Solver}.

The \texttt{Servlet} class extends the standard \texttt{HttpServlet} class and
provides common functionality. The \texttt{Servlet} class provides a base for 
all system Servlets to use the common functionality.

The \texttt{Servlet} class is able to if a given request is from a JSON, XML or 
Ajax background. For example if a client was to make a request to the web 
service through a web browser utilising an Ajax request, then the 
\texttt{isAjaxRequest}  method would return \texttt{true}.

For illustrative proposes the \texttt{isAjaxRequest} method is shown below.

\begin{verbatim}
  protected boolean isAjaxRequest(HttpServletRequest request) {
    String ajax = request.getHeader("x-requested-with");
    return ajax != null && ajax.toLowerCase().contains("xmlhttprequest");
  }
\end{verbatim}

The servlet also contains two customised methods -- one to handle errors, and 
the other to handle a good response -- that are able to send a response back to 
the requesting client based upon a number of factors.

For example the methods are able to convert the return data into either XML or 
JSON depending upon what the client has asked for. The \texttt{sendError} method
will also set the HTTP status code correctly, allowing the client to correctly 
authenticate the response.

Finally the \texttt{Servlet} class overrides the \texttt{init} method, which 
is automatically called as part of the object construction. This method will 
initialise resources at servlet creation (i.e. first run-time within tomcat) 
rather than during the first call to the servlet. 

In doing this, Tomcat will take more time initially starting up, however the 
user will notice that their queries are dealt with much quicker. In this project
the \texttt{init} method has been used to initialise the various in-memory 
dictionaries and thesauri.

~\\

The second class is the \texttt{Solver} servlet and handles all request that are
specifically for solving a given clue. The \texttt{Solver} servlet accepts both
GET and POST requests, with each requiring the clue, the length of the solution
and the solution pattern.

The \texttt{Solver} servlet upon receiving a request will validate the input 
parameters, based upon a number of criteria including presence checks and 
regular expressions. The code snippet below is an example validation rule, that 
will validate the solution pattern against a regular expression.

\begin{verbatim}
 private boolean isPatternValid(String pattern) {
    // Pattern string regular expression
    final String regex = "[0-9A-Za-z?]+((,|-)[0-9A-Za-z?]+)*";
    boolean match = Pattern.matches(regex, pattern);

    // Pattern String must be present and of a valid format
    return isPresent(pattern) && match;
  }
\end{verbatim}

In order for validation to pass, the solution pattern must not be empty and must
match to the regular expression stated in the method.

The \texttt{Solver} servlet class will initialise the solving of a clue if the 
three inputs are deemed to be valid. The \texttt{solveClue} method will utilise 
the Clue manager class, that will handle the distributing of the clue to the 
various solvers. 

This has been designed so that the servlet and the solving processes are upon 
separate threads. This prevents tomcat from freezing, and allows it to handle 
requests from users.

Once all the solvers have finished executing, the \texttt{Solver} servlet will 
produce an XML document based upon the various elements. Once the XML document 
has been created, will be sent back to the client as either XML or JSON.
