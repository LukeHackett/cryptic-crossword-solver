%%%
%% Development :: Core
%%%
\section{Core}
\label{sec:core}

Within this section the core package will be presented. The core package is the 
package that forms the heart of the system, and provides various base classes
that when instantiated will represent Clues, Solutions and Solution Patterns.


%%%
%% Development :: Core :: Clue
%%%
\subsection{Clue}
\label{sub:clue}

The clue class represents an individual cryptic crossword clue, and will 
maintain a list of possible solutions that have been computed. The Clue class is
a basic class, that essentially provides references to other aspects of solving 
a clue, such as the solution pattern (see subsection \ref{sub:solution_pattern}).

A clue is able to have a number of solutions, and hence each clue will house a 
SolutionsCollection, which is described in more detail in subsection 
\ref{sub:solution_collection}.

A clue will also have a solution pattern -- described in subsection 
\ref{sub:solution_pattern} -- which allows for potential solutions to be matched 
against an expected pattern.


%%%
%% Development :: Core :: Solution
%%%
\subsection{Solution}
\label{sub:solution}

The solution class implements the Comparable interface, which is a standard Java
interface that imposes ordering upon the object that implements it. It was 
decided that the solution class should implement the interface, so that 
solutions can be compared.

The comparing of solutions is perhaps one of the most common pieces of 
functionality that the system will be required to do. An example use case would
be comparing any number of solutions to deduce which is more likely to be 
correct answer.

Listing \ref{compareToSolution} shows the implemented compareTo() method found
within the Solution class.

\begin{lstlisting}[caption={compareTo() compares two solutions},
                   label=compareToSolution] 
  public int compareTo(Solution o) {
    int solutionCompare = solution.compareTo(o.getSolution());
    int confidenceCompare = -1
        * Double.compare(confidence, o.getConfidence());

    if (solutionCompare == 0) {
      // compare the actual solution text.
      return 0;
    } else if (confidenceCompare == 0) {
      // return a comparison of the solution text. 
      return solutionCompare;
    } else {
      // compare them based on their confidence.
      return confidenceCompare;
    }
  }
\end{lstlisting}

The above code will try to deduce which of the two solutions are closest to 
being correct. It does this by comparing their confidence ratings, to which 
every solution will have. If the solution is the same the 0 is returned, whist 
-1 or +1 returned depending upon which solution is closest to being correct.

As well as the housing the confidence rating, the solution class also houses the
solution trace. The solution trace is a List of steps that were taken in order 
to compute the solution. It is intended that the solution traces will help to 
teach the user how to complete similar clues in the future. 


%%%
%% Development :: Core :: Solution
%%%
\subsection{SolutionCollection}
\label{sub:solution_collection}

The SolutionCollection class extends the standard Java HashSet class utilising 
the Solution class as the element type. Although a HashSet can not guarantee the
order of the set, it can guarantee that no duplicates will be added to the set.
This decision was taken to ensure that the system is not dealing with large 
amounts of repetitive datasets, that will only harm the performance of the 
system as whole.

In order to get around the fact that the ordering of the set has not guarantee,
the system makes use the of fact that the element type -- Solution -- implements
the Comparable interface. This means that a copy of the current collection can 
be retuned in a sorted order if possible. This is illustrated in listing 
\ref{sortSolutions_SolutionsCollection}.

\begin{lstlisting}[caption={Method returns a new sorted collection},
                   label=sortSolutions_SolutionsCollection] 
  public Set<Solution> sortSolutions() {
    return new TreeSet<>(this);
  }
\end{lstlisting}

As both the TreeSet class and the HashSet both share the same parent class Set,
it is possible to cast the result of this method back to a SolutionCollection.

The SolutionCollection class overrides basic methods found within the HashSet 
class, such as \texttt{contains}, \texttt{add}, \texttt{addAll}, whilst 
providing additional functionalities, such as returning all solutions whose 
confidences are greater than (or less than) a given value.


%%%
%% Development :: Core :: Solution Pattern
%%%
\subsection{Solution Pattern}
\label{sub:solution_pattern}

The SolutionPattern class models the solution to a corresponding clue. For 
example if the clue is nine letters long, then the SolutionPattern class would 
provide information about the pattern of that solution using any given known 
characters.

The solution pattern is able to split a well-formed pattern and represent it 
as an in-memory object allowing for a faster clue matching process, in 
comparison to constantly trying to match a string.

An good example of the level of functionality available in this class is shown 
in listing \ref{filter_SolutionsPattern}.

\begin{lstlisting}[caption={Method returns collection of matched solutions},
                   label=filter_SolutionsPattern] 
public void filterSolutions(Set<Solution> solutions) {
  Collection<Solution> toRemove = new ArrayList<>();
  // For each proposed solution
  for (Solution solution : solutions) {
    // If it doesn't match the pattern, throw it out
    if (!match(solution.getSolution())) {
      toRemove.add(solution);
    }
  }
  solutions.removeAll(toRemove);
}
\end{lstlisting}  

The code snippet will match the given set of solutions --- often a 
SolutionCollection --- to the current object. A match can simply be described as
matching a solution pattern to a possible solution, for example `d??k' could 
match to `duck', `deck' or even `dork'.

This method is often used within the filtering down of potential solutions, and
thus ensures that the application is not dealing with too much data. In effect
this helps the application become more efficient when solving solutions. 

Obviously the efficiency of the matching process is directly linked to the 
number of known characters. If large number of known characters are given by the
user, then the total `search space' is dramatically reduced.

As an example there are about 308 million different letter combinations within a
six letter word ($26^6$) -- this refers to the arrangement of letters, and not 
`actual' words. If just one of those letters is known, this would be reduced 
down to 11 million different letter combinations ($26^5$), and knowing two 
letters would reduce it down to around half a million ($26^4$).


%%%
%% Development :: Core :: Manager
%%%
\subsection{Manager}
\label{sub:clue}

The Manager class manages the process of solving the given clue. The manager 
class heavily utilises the standard Java future interface, which is designed to
represents the result of asynchronous computation.

Listing \ref{distributeAndSolveClue} illustrates the distribution functionality 
that distributes a copy of all the necessary resources --- such as the clue 
and it's solution pattern -- and starts the computation upon various new 
threads (if available upon the system).

Each of the solvers are obtained dynamically via the plug and play system to 
which more information can be found within section 
\ref{sec:plug_and_play_architecture} on page \pageref{sec:plug_and_play_architecture}.
Once the solvers are obtained they are initialised. The initialisation process
simply creates a new object instance of the solvers, and starts them solving the
clue.

Once all solvers have finished, their computed solutions are added to an overall
solutions collection, which will ignore duplicate solutions as explained in 
section \ref{sub:solution_collection}.

Each of the solutions will have their confidences adjusted based upon how 
`correct' the solution is.

\begin{lstlisting}[caption={Method to distrbute the clue out to all solvers},
                   label=distributeAndSolveClue] 
public SolutionCollection distributeAndSolveClue(Clue clue) {
    // This will hold the solvers to be run at runtime
    Collection<Solver> solvers = getSolversFromClasses(clue);
    // This will hold the returned data from the solvers
    Collection<Future<SolutionCollection>> solutions = initiateSolvers(solvers);
    // This will hold all solutions that have been returned
    SolutionCollection allSolutions = new SolutionCollection();
    // Now we need to 'unpack' the SolutionCollections
    for (Future<SolutionCollection> future : solutions) {
      try {
        allSolutions.addAll(future.get());
      } catch (InterruptedException | ExecutionException e) {
        e.printStackTrace();
      }
    }
    // Adjust confidence scores based on category matches
    Categoriser.getInstance().confidenceAdjust(clue, allSolutions);

    return allSolutions;
  }
\end{lstlisting}  
