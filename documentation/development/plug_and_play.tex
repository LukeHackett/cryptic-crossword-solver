%%%
%% Development :: Plug and Play Architecture
%%%
\section{Plug and Play Architecture}
\label{sec:plug_and_play_architecture}

One of the major aspects of the design and development of the system was the 
underlying architecture. A well designed architecture would not only improve the
overall efficiency of the system, it would also aid in reducing development time
and improve the maintainability of the system.

It was decided that a `plug and play' architecture would be adopted. A plug and
play architecture allows for external components to be attached to a main 
system, and thus allows for that system to be able to detect that component and 
use it \citep{plug_play}.

The plug and play architecture is a common model found within computing, an 
example use age would be a network. A computing device is able to connect 
(either via a wire or wireless) to a network and it will automatically be able 
to access the network resources without having to reconfigure the computer, the 
network or other devices connected to that network.

The plug and play architecture would aim to solve the problem of having an 
unknown number of solvers working within the overall system. By implementing the
architecture, any number of solvers could be added and removed to the system at
any time (compile time or run time). Thus allowing the system to be as flexible 
as possible when it comes to handling the solvers. As a bi-product of this, it 
would also enable any future developers to be able to add their own solver 
without having to re-factor any core application code.

The architecture uses an `includes' file which lists all solvers that are 
intended to be included within the system. The file also supports the ability to
ignore files which enables solvers to not be included in the system at run time.
Listing \ref{solverImports} illustrates an example snippet. Lines that start 
with `\#' are files that are to be ignored --- i.e. the class will not be 
imported at run time. This is illustrated in listing \ref{solverImports}.

\begin{lstlisting}[caption={plug and play solver import file}, label=solverImports]  
uk.ac.hud.cryptic.solver.Acrostic
uk.ac.hud.cryptic.solver.Anagram
#uk.ac.hud.cryptic.solver.Homophone
#uk.ac.hud.cryptic.solver.DoubleDefinition
uk.ac.hud.cryptic.solver.Pattern
\end{lstlisting}

Listing \ref{solverImports} illustrates that the Acrostic, Anagram and Pattern 
solvers will be `loaded' in to the system's list of available solvers. Whilst 
the Homophone and DoubleDefinition solvers will not be loaded. It must be stated
that currently the system only supports the loading of solvers at the initial 
start up of the application. Solvers can not be added or removed during the use 
of the application, but due to the flexibility of the architecture this could be
easily added in a future revision.

The loading and parsing of the classes are handled within the manager class 
found within the core package, which was described in more in detail in section
\ref{sub:manager} on page \pageref{sub:manager}.

Each of the required classes added to the properties file are imported by 
utilising Reflection. Reflection allows for the solver classes to be `loaded' at
run time, by importing the Java class and then instantiating it at run time. 
Listing \ref{loadSolver} illustrates the snippet of code that `imports' and 
initialises the various solvers.

\begin{lstlisting}[caption={run time class import using reflection}, label=loadSolver]  
try {
  // Name of the solver read from the propertes file
  String solverName;
  // Load the class
  Class<?> cls = Class.forName(solverName);
  Constructor<?> c = cls.getDeclaredConstructor(Clue.class);
  // Instantiate the given solver class
  Solver solver = (Solver) c.newInstance(clue);
  // Add to the list of solvers
  solvers.add(solver);
} catch (Exception e) {
  // Error handling omitted
}
\end{lstlisting}

Although reflection may provide a number of issues such as security implications
and performance issues in this instance it was deemed to be a benefit, as it 
allows for extensible features (solvers) to be defined at run time, rather than
at compile time.

As long as all additional solvers extend the main solver class, then the above 
reflection practise will work for all future solvers.
