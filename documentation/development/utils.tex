%%%
%% Development :: Utilities
%%%
\section{Utilities}
\label{sec:utilities}

The utilities package (named utils) is a package that contains various classes 
that cannot be grouped together in their own package, and provide generic helper
methods in the aiding the system as a whole.

Within this section two commonly used classes --- Confidence and WordUtils --- 
will be discussed in more detail.


%%%
%% Development :: Utilities :: Confidence
%%%
\subsection{Confidence}
\label{sub:confidence}

The confidence class will manage the various functions that are associated with
calculating and assigning confidence scores. The calculating of any given 
confidence score can be easily achieved across multiple solvers by focusing upon
the `common elements' that are found between all solvers. Once the `common 
elements' are calculated then specifics ratings can be applied on top.

Each generated solution will start with an initial confidence of 42. The 
confidence rating will then be increased or deceased based upon certain features
that are found within the clue, solution and the solver that managed to generate
the solution.

As previously mentioned there are a number of features, and to describe all of 
them would be laborious, however three example features will be presented 
below:

\begin{itemize}
  \item if a solution is a synonym of the clue's definition word then the 
        confidence is increased by 50\%
  \item if a solution has a synonym of the clue's definition word then the 
        confidence is increased by 10\%
  \item if a double-definition with two first-level synonyms is found then the
        confidence rating is increased by 40\%
\end{itemize}

It is possible for more than one confidence increase to be applied to one 
solution, however it is not possible for a confidence to be less than 0 or 
greater than 100.

The confidence class is a `bear bones' class that simply stores the value of
each type of confidence increase, and by how much that increase should be. The 
confidences adjustments are often made by the various resources as shown in the
\nameref{sec:resources} section.


%%%
%% Development :: Utilities :: WordUtils
%%%
\subsection{WordUtils}
\label{sub:wordutils}

The WordUtils class is a collection of helper methods relating to the 
manipulation of words and sentences. The class is a static class, and has been 
set as static due to the fact that the WordUtil functionality is used 
extensively throughout the system.

An example method is shown in listing \ref{hasCharacters}. This method will 
deduce if a word can be `created' using the supplied string of characters. This
method is often used in conjunction with the dictionary and thesaurus classes
when trying to find hidden words.

\begin{lstlisting}[caption={deduces if a word can be made with a given set of 
                            unilearncharacters}, label=hasCharacters] 
public static boolean hasCharacters(String targetWord, String characters) {
  // This list will be reduced as characters are consumed
  Collection<Character> remaining = new ArrayList<>();
  for (char c : characters.toCharArray()) {
    remaining.add(c);
  }
  // Loop over each of the target word's characters
  for (char c : targetWord.toCharArray()) {
    // Continue if the char is available in the character pool
    if (!remaining.remove(c)) {
      return false;
    }
  }
  // the target word can be built
  return true;
}
\end{lstlisting}
