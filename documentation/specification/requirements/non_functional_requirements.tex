%%%
%% Specification :: Non-Functional Requirements
%%%
\section{Non-Functional Requirements}


%%%
%% Specification :: Non-Functional Requirements :: Performance
%%%
\subsection{Performance}

The performance of the system should be good, however it must be noted that the
system is intended to run as a web service upon a server. The quality of the 
connection between the end user and the web service will be out of the scope of
this project.


%%%
%% Specification :: Non-Functional Requirements :: Legal and Access
%%%
\subsection{Legal and Access}

Within the project a number of data sources will be used. Firstly an English
dictionary will need to be accessed in order to allow the system to deduce if a 
given arrangement of characters formulates an English word.

In order to achieve this, an offline dictionary and an online dictionary will
need to be used. The offline dictionary is a short dictionary and is provided as
part of the GNU/Linux operating system under the GNU Licence Agreement.

A number of cryptic crossword clues and solutions have been obtained from the
Guardian's website. The data will be used as a training set of data, when
testing the final software. The Guardian has given us permission to use their
data as long as the project remains within an academic environment, and that no
profit is made from the project.


%%%
%% Specification :: Non-Functional Requirements :: Backup and Recovery
%%%
\subsection{Backup and Recovery}

The project will make use of the git revision control system over a secure shell 
connection. The service provider is github, who offer a distributed setup to 
ensure that data is always available. Using a secure revision control system 
will ensure that:

\begin{enumerate}
  \item A copy of the project is stored upon a remote, secure server;
  \item Changes are able to be tracked;
  \item Issues and comments are able to be raised in a secure environment.
\end{enumerate}

The above measures will in effect be able to reduce against the loss, damage or 
theft of all project information.


%%%
%% Specification :: Non-Functional Requirements :: Archiving and Retention
%%%
\subsection{Archiving and Retention}

The University will retain a copy of the project (software product and the final
report) for assessment and evaluation purposes. 


%%%
%% Specification :: Non-Functional Requirements :: Maintainability
%%%
\subsection{Maintainability}

The project -- including the software product -- will be maintained up until the
project hand in deadline: 9th May 2014.


%%% 
%% Specification :: Non-Functional Requirements :: Availability
%%%
\subsection{Availability}

The software will not contain any request or time limitations for users, and 
thus should be available at all times. However it must be stated that the server
availability that will be hosting the web service is out of the scope of the 
project.


%%%
%% Specification :: Non-Functional Requirements :: Capacity
%%%
\subsection{Capacity}

The end product should be able to handle a number of requests from different 
users at the same time. As the project is only intended as an academic project, 
large scale user support will be out of the scope of this project.


%%%
%% Specification :: Non-Functional Requirements :: Usability Requirements
%%%
\subsection{Usability Requirements}

\begin{table}[H]
  \centering
  \small
    \begin{tabular}{|p{9.3cm}|p{1.3cm}|p{1.3cm}|p{1.3cm}|p{1.3cm}|}
    \hline
    \textbf{Functionality} & \textbf{Must} & \textbf{Should} & \textbf{Could} & \textbf{Won't} \\ \hline

    Provide a text field for the user to input the cryptic clue to be solved &
    Yes & - & - & - \\ \hline

    Provide a drop-down box allowing the user to input the number of words in 
    the solution, with an initial upper-limit of 10 words &
    Yes & - & - & - \\ \hline

    Dynamically provide individual text boxes for each word of the solution, 
    which allow the length of the words to be specified. Also provide 
    check-boxes between these text-boxes to define whether they are separated by
     a space or a hyphen &
    Yes & - & - & - \\ \hline

    Dynamically provide text boxes to represent each character of each word of 
    the solution, allowing the user to input any known characters &
    Yes & - & - & - \\ \hline

    Provide a table of results which allow the user the to view the possible 
    solutions &
    Yes & - & - & - \\ \hline

    Alert the user to required fields with red asterisks &
    Yes & - & - & - \\ \hline

    Have a consistent layout avoid unnecessary scrolling &
    - & Yes & - & - \\ \hline

    Alert the user to invalid input through validation checks with error 
    messages &
    - & Yes & - & - \\ \hline

    Provide a confidence rating in the table of possible solutions &
    - & Yes & - & - \\ \hline

    Allow the user to select a proposed solution in the corresponding table and 
    mark this as correct &
    - & Yes & - & - \\ \hline

    Display help buttons to indicate to the user the purpose and use of each 
    control &
    - & - & Yes & - \\ \hline

    Provide a button to submit the user input to the application for processing &
    - & - & Yes & - \\ \hline

    Provide a group of radio buttons for the user to select the clue's orientation
    in its containing crossword &
    - & - & Yes & - \\ \hline

      Provide a text box to input the clue's number within its containing 
      crossword &
    - & - & Yes & - \\ \hline

    Provide mechanisms to accommodate for users with difficulties, such as 
    colour-blindness or poor eyesight &
    - & - & Yes & - \\ \hline

    \end{tabular}
    \caption {MoSCoW analysis of the project's usability requirements}
\end{table}