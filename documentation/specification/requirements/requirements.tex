\chapter{Software Specification} 
\label{chap:requirements} 

\citet{cadle10} states that there is a standard hierarchical approach in
structuring requirements. Table \ref{table:requirementsCategories} outlines the
main four categories:

\begin{table}[H]   
  \begin{tabular}{|l|l|l|l|}
  \hline     
  {\bf General}        & {\bf Technical}  & {\bf Functional} & {\bf Non-Functional}    \\ 
  \hline
  Business constraints & Hardware         & Data entry       & Performance             \\      
  Business policies    & Software         & Data maintenance & Security                \\      
  Legal                & Interoperability & Procedure        & Legal and Access        \\      
  Branding             & Internet         & Retrieval        & Backup and Recovery     \\      
  Cultural             & ~                & ~                & Archiving and Retention \\      
  Language             & ~                & ~                & Maintainability         \\      
  ~                    & ~                & ~                & Business Continuity     \\      
  ~                    & ~                & ~                & Availability            \\      
  ~                    & ~                & ~                & Usability               \\      
  ~                    & ~                & ~                & Capacity                \\     
  \hline   
  \end{tabular}
  \label{table:requirementsCategories} 
\end{table}


Although these are some of the sections usually found in the structure of
requirements there are several books which state that each section depends on
the type of project being undertaken. 

\citet{robertson13} mentions that `a requirement is something the product must 
do to support its owners business, or a quality it must have to make it 
acceptable and attractive to the owner.' The aim of gathering the requirements 
has always been to ensure that all ambiguities are removed before a product is 
developed. 

As well as the process of requirements engineering there are standards available
for organizations to adhere to when compiling their documents. A more common
standard is the ISO 9001 for Quality Assurance. In the document it is described
that an organizations system is influenced by:

\begin{itemize}
  \item its organizational environment, changes in that environment, and the 
        risks associated with that environment
  \item its varying needs
  \item its particular objectives
  \item the products it provides
  \item the processes it employs
  \item its size and organizational structure
\end{itemize}
\hfill\citep{iso08}

As the project consists of four members whom are part of all development teams 
such standards do not need to be followed but an awareness of these standards is
beneficial. The advantage of a standard such as the ISO 9001 would be to the
project would be only if the project is integrated as a business outside the
scope of University studies.

The following sections have been derived from the book by \citet{robertson13}.
The book contains a very useful template called the Volere template which aims
to recognize as much information required when engineering requirements.


% Aims and Objectives
\newpage
%%%
%% Specification :: Aims & Objectives ::
%%%
\section{Aims \& Objectives}


%%%
%% Specification :: Aims & Objectives :: Aims
%%%
\subsection{Aims}

\begin{itemize}
  \item Produce a software application that accepts the input of cryptic 
        crossword clues from a user
  \item Produce a software application that will present a set of results to a
        user for the cryptic crossword clue which has been input
  \item Effectively adhere to a project plan as a team
  \item Effectively abide by the guidelines of a chosen software methodology
  \item Construct systematic and comprehensive documentation for the academic
 report
\end{itemize}


%%%
%% Specification :: Aims & Objectives :: Objectives
%%%
\subsection{Objectives}

\begin{itemize}
  \item To research cryptic crosswords themselves and the various types of 
        cryptic crossword clue
  \item To study relevant software architectures associated with the group 
        project
  \item To investigate the field of artificial intelligence, in particular 
        natural language processing and justify the need for it within the 
        project to solve cryptic crossword clues
  \item To research the various software methodologies which exist and select 
        the most appropriate for the project
  \item To understand and effectively follow a collection of requirements
  \item Evaluate the progress of the project at numerous instances throughout 
        the life cycle of the project
\end{itemize}

% Project Constraints % Mandated Constraints 
\newpage
\newpage
\section{Mandated Constraints}

The following section describes the constraints that effect the design of the
product. The product that is to be developed cannot be successful unless these
constraints have been accomplished.

\section{Solution Constraints}

\noindent\llap{\textbf{[R1/1]}}The product shall be built for the following platforms:\\
\begin{enumerate}
		\item Blackberry
		\item iOS 
		\item Android
\end{enumerate}

\textbf{Rationale:}  The product it to be able to be used on the go.\\
\textbf{Volatility:} High


\noindent\llap{\textbf{[R2/1]}}The product shall require an Internet connection. \\

\textbf{Rationale:}  The product cannot work without an Internet connection.\\
\textbf{Volatility:} High


% Relevant_Facts_And_Assumptions 
\newpage
\section{Relevant Facts and Assumptions}

\subsection{Facts}

\noindent\llap{\textbf{[I1/1]}}Answers to Cryptic Crosswords are usually published the following day

\noindent\llap{\textbf{[I2/1]}}The same clues does not always have the same answers

\noindent\llap{\textbf{[I3/1]}}Existing applications don't offer real time solvers

\noindent\llap{\textbf{[I4/1]}}Electronic solvers not available

\subsection{Assumptions}

\begin{enumerate}
\item Equipment needed to test the application on different platforms will be
available.
\item A server to host the web service will be available.
\item The Guardian has given permission to use their cryptic crossword data to use for
our test data.
\item Four team members working on the project throughout the year at approximately
25\% of their academic study time
\item Project scope will remain the same throughout the project
\end{enumerate}

% Scope
\newpage
%%%
%% Specification :: Scope
%%%
\section{Scope}

%%%
%% Specification :: Scope :: Organisation Structure
%%%
\subsection{Organisation Structure}

The project team is based largely upon democratic discussions and decisions,
however to ensure that team deadlocks do not occur a project leader has been
chosen. Figure ~\ref{fig:org_hierachy} reflects the hierarchy of the team.

\begin{figure}[H]
  \centering
  \includegraphics[width=0.9\textwidth]{organisation_structure.png}
  \caption{Hierarchical Structure of the team}
  \label{fig:org_hierachy}
\end{figure}


%%%
%% Specification :: Scope :: Organisation Structure :: Methodology
%%%
\subsection{Methodology}

It has been decided that the team will follow an Agile software development
model. This will allow the team to split the larger task down into smaller, more
manageable `chunks', that allows for a good quality analysis, evaluation,
development and planning (on to the next `chunk'). An iterative approach is best
suited for this project due to the nature of changes and updates the product
will require in future builds.

This method of development also allows for a more feature-driven approach to the
project. Ultimately this allows for more important features and aspects of the
project to be completed first.


%%%
%% Specification :: Scope :: Meetings
%%%
\subsection{Meetings}

A weekly meeting will take place between all project members to discuss all
aspects of the project. This includes (but is not limited to) project issues,
software development issues, research findings, possible improvements and code
reviews.


%%%
%% Specification :: Scope :: Product
%%%
\subsection{Product}

Within the cryptic crossword area, there is a need for a cryptic crossword clue
solver in the form of a software application.  This has been justified
previously within the document through research and investigating existing
applications within the field. The main purposes of the software deliverable are
to allow the input of a cryptic crossword clue by a user and output a potential 
results or number of results.

Once the project has been completed by the team, the following deliverables will
have been accomplished:

\begin{itemize}
  \item A software application which accepts input from the user and outputs 
        appropriate solutions. 
  \item Two written reports that:
    \begin{itemize}
      \item document the entire software development process from a group 
            perspective
      \item analyse and evaluate the project as a whole
    \end{itemize}
\end{itemize}

Furthermore the subsequent internal components will be implemented to aid with 
the goals of the project:

\begin{itemize}
  \item A storage area to collect and store the cryptic crossword clues and 
        their details used for test data as well as, clues successfully solved 
        by the application which a user has input.
  \item A service which will allows connections from web browsers and 
        potentially mobile phones to ensure the running of the software 
        application.
\end{itemize}

There are specific criteria which have been identified as important features
which must be completed to deem the project as adequately finished:

\begin{itemize}
  \item A document outlining the full software development life cycle of the 
        project
  \item A software application which can be accessed via either:
    \begin{itemize}
      \item A web browser \textbf{and/or}
      \item A portable device (e.g. smartphone or tablet)
    \end{itemize}
  \item A service which can be connected to by either a mobile phone or a web 
        browser which will sufficiently solve a cryptic crossword clue and 
        output a result or number of results 
  \item A storage area accessible by the service which stores the cryptic 
        crossword clues and their associated details
\end{itemize}

Furthermore, there are specific criteria which have been deemed as out of the 
scope of the project:

\begin{itemize}
  \item The user will not be able to access the storage area to browse through 
        data
  \item The software application will not be implemented to have the ability to
        generate cryptic crosswords from the data stored
\end{itemize}

The restrictions listed below are the justifications for the project scope:

\begin{itemize}
  \item The time scale of the project
  \item External priorities from the academic year such as other module 
        assignments and examinations
  \item Limited resources for testing the software application on restrict the
        platforms which the software application will be fully compatible with
\end{itemize}


% Functional Requirements 
\newpage
\section{Functional Requirements}

These are things the product must do.

e.g The product shall produce a result upon it given a clue within 30 seconds.


\subsection{Data entry}

\subsection{Data maintenance}

\subsection{Procedure}

\subsection{Retrieval}



% Non-functional Requirements 
\newpage
\section{Non-Functional Requirements}

\begin{table}[H]
	\centering
	\small
    \begin{tabular}{|p{9.3cm}|p{1.3cm}|p{1.3cm}|p{1.3cm}|p{1.3cm}|}
    \hline
    \textbf{Functionality} & \textbf{Must} & \textbf{Should} & \textbf{Could} & \textbf{Won't} \\ \hline

    Provide a text field for the user to input the cryptic clue to be solved &
    Yes & - & - & - \\ \hline

    Provide a drop-down box allowing the user to input the number of words in the solution, with an initial upper-limit of 10 words &
    Yes & - & - & - \\ \hline

    Dynamically provide individual text boxes for each word of the solution, which allow the length of the words to be specified. Also provide check-boxes between these text-boxes to define whether they are separated by a space or a hyphen &
    Yes & - & - & - \\ \hline

    Dynamically provide text boxes to represent each character of each word of the solution, allowing the user to input any known characters &
    Yes & - & - & - \\ \hline

    Provide a table of results which allow the user the to view the possible solutions &
    Yes & - & - & - \\ \hline

    Alert the user to required fields with red asterisks &
    Yes & - & - & - \\ \hline

    Have a consistent layout avoid unnecessary scrolling &
    - & Yes & - & - \\ \hline

    Alert the user to invalid input through validation checks with error messages &
    - & Yes & - & - \\ \hline

    Provide a confidence rating in the table of possible solutions &
    - & Yes & - & - \\ \hline

    Allow the user to select a proposed solution in the corresponding table and mark this as correct &
    - & Yes & - & - \\ \hline

    Display help buttons to indicate to the user the purpose and use of each control &
    - & - & Yes & - \\ \hline

    Provide a button to submit the user input to the application for processing &
	- & - & Yes & - \\ \hline

	Provide a group of radio buttons for the user to select the clue's orientation in its containing crossword &
	- & - & Yes & - \\ \hline

    Provide a text box to input the clue's number within its containing crossword &
	- & - & Yes & - \\ \hline

	Provide mechanisms to accommodate for users with difficulties, such as colour-blindness or poor eyesight &
	- & - & Yes & - \\ \hline

    \end{tabular}
    \caption {MoSCoW analysis of the project's non-functional requirements}
\end{table}

\section{Usability and Humanity Requirements}

%Risks 
\newpage 
\section{Risk Assessment}

% Introduction.
An investigation into the potential risks that may present themselves during the course of this project is of paramount importance if their impact is going to be minimised, should they become a reality.

% Included here a risk impact table
\subsection{Risk Identification}

% PI Scores and risk classification graph
\subsection{Risk Classification}

\subsection{Risk Prevention}

\subsection{Risk Mitigation}

%Feasibility Study
\newpage
\section{Feasibility Study}

A feasibility study will be carried out to identify the difficulty of the
 implementation needed for each type of cryptic crossword clue with
 the aid of the research carried out previously. The study will focus
 on the resources needed for each implementation which will contribute
 to the level of difficulty for each clue type. Furthermore, using the 
previous research resources within the cryptic crossword field, a measurement
 of how common each clue type is within a cryptic crossword will be presented
 to assist in the understanding of the necessity of the clue type within the project.

Aspects such as how common clue types are featured within research
 resources and how regular indicators are found in the project test data
 will contribute to the judgement made on how regular a type of clue is. The
 justification of the use of specific resources for a clue type will come from
 research resources and inspecting project test data.

\begin{table}[H]
	\centering
	\small
    \begin{tabular}{|p{4cm}|p{4cm}|p{4cm}|p{4cm}|}
    \hline
    \textbf{Clue Type} & \textbf{Possible Resources} & \textbf{Difficulty} & \textbf{Regularity} \\ \hline

     	 Hidden & Dictionary & Low & Common \\ \hline

     	 Anagrams & Dictionary & Low & Common \\ \hline	
	 
	Acrostics & Dictionary & Low & Intermediate \\ \hline
 
	Pattern & Dictionary & Low & Intermediate \\ \hline

           Homophones & Homonym Dictionary & Medium & Common \\ \hline

	Charades & Abbreviations, Dictionary, Thesaurus & Medium & Common \\ \hline

	Deletions & Abbreviations, Dictionary, Thesaurus & Medium & Common \\ \hline

	Reversals & Abbreviations, Thesaurus & Medium & Common \\ \hline
 
	Palindromes & Dictionary, Thesaurus & Medium & Intermediate \\ \hline
 	
	Double Definition & Dictionary, Thesaurus & Medium & Intermediate \\ \hline
 
	Substitutions & Abbreviations, Dictionary & Medium & Rare \\ \hline
 	
	Shifting & Dictionary, Thesaurus & Medium & Rare \\ \hline
 	
	Exchange & Dictionary, Thesaurus & Medium & Rare \\ \hline
 
	Spoonerisms & Dictionary, Thesaurus & Medium & Rare \\ \hline
 
	Containers & Abbreviations, Dictionary, Thesaurus & High & Common \\ \hline
 
	Purely Cryptic & - & High & Common \\ \hline

	\& lit & Abbreviations, Dictionary, Thesaurus & High & Intermediate \\ \hline

    \end{tabular}
    \caption {Feasibility Study for Clue Types}
\end{table}

