\chapter{Development Methodologies}
\label{cha:development_methods}

In order to ensure all objectives and goals that have been set within this 
project are completed to the highest quality and upon time, a software 
development methodology will need to be chosen. Dividing a larger project into 
a set number of defined processes may seem like additional unnecessary work, 
but the advantages of this process far outweigh the disadvantages 
\citep{knott_dawson99}.

The defined processes combine together to form part of a process model. The 
process model will allow for the following achievements \citep{knott_dawson99}:

\begin{itemize}
  \item Adding an element of control and planning
  \item Allowing for progress to be mapped visually
  \item Providing a structured approach to development
  \item Allowing for a higher quality of code and documentation to be produced
\end{itemize}

The Systems Development Life Cycle (SDLC) was one of the first formalised 
methodologies for building software. The SDLC utilises a methodical and 
structured approach to analysing, designing, building and testing software, to 
which many methodologies follow this rigid structure \citep{elliott04}. 


% Waterfall overview 
\newpage
\section{Waterfall}

One of the main aspects to the waterfall model is the fact that the project is 
expected to progress down the primary path \citep{cadle10}.

The waterfall model takes the major components of any project (requirements, 
design, implementation, testing and maintenance) and assigns each component a 
stage of its own. Each component is delivered as the flow down the primary path
is completed \citep{cadle10}.

The waterfall model also supports backtracking (i.e. reverting back to a 
previous deliverable). This allows for project managers to check that the 
project has not expanded its defined scope, and to also ensure that each 
deliverable flows into the next correctly. It also allows for slight 
modifications to be made, however making many large changes might affect the 
project in the long run \citep{cadle10}.


\subsection{Advantages}
\citet{cadle10} states that the waterfall model houses a number of advantages, 
including:

\begin{itemize}
  \item Provides a rigid project structure, that is easy to follow and review
  \item Deliverables are delivered in project order, one at a time
  \item Can work well for smaller projects, or for projects where by the 
  requirements will not change. 
\end{itemize}


\subsection{Disadvantages}
However \citet{cadle10} also goes on to state that the waterfall model houses a
number of potential problems, including:

\begin{itemize}
  \item Changes are difficult to implement the further a project is down its 
  primary path
  \item Large projects may not benefit from the rigid structure
  \item A working piece of software is not delivered until late into the 
  project
\end{itemize}



% Spiral overview 
\newpage
\section{Spiral}

A common feature found in the waterfall model is that all requirements are 
stated at the start of the project. It is these requirements that will form the
basis of all work, along with any project planning \citep{cadle10}.

The spiral model forms its basis around iteration and prototyping to try to 
explore the requirements and develop the solution. During each turn around the
spiral, a set of requirements are analysed and developed using prototyping 
\citep{cadle10}.


\subsection{Advantages}
\citet{cadle10} states that the spiral model houses a number of advantages, 
including:

\begin{itemize}
  \item A high amount of risk analysis is conducted, and thus risk is more 
  likely to be avoided
  \item The model allows for approval from clients, and large amounts of 
  documentation to be produced
  \item Software can start to be produced earlier, in comparison to the 
  waterfall methodology
  \item Additional functionality can be added on at any time during or after 
  the project
\end{itemize}


\subsection{Disadvantages}
The spiral model allows for a high level of control, without too much 
restriction. However \citet{cadle10} states that this can cause difficulties 
such as:

\begin{itemize}
  \item A thorough investigation into all of the requirements cannot be 
  achieved early, therefore some requirements (and their priorities) may get 
  completely missed
  \item The spiral model is based upon the clients knowing exactly what they 
  want, which is unlikely
  \item A risk analysis must be conducted, and requires highly specific 
  expertise to complete. If a risk analysis is not completed, then the project
  may completely fail 
\end{itemize}



% Agile overview 
\newpage
\section{Agile}

The agile software development methodology is designed to ``reduce risk by 
delivering software systems in short bursts or releases'' \citep{dawson09}.

Each release (sometimes referred to as iterations) will involve minimal 
planning and will cover all the major SDLC components: analysis, design, 
implementation and testing. The agile development model also heavily promotes 
collaboration/development between team members \citep{dawson09}. 


\subsection{Advantages}

One of the main advantages of using the agile development model is that 
software is developed in rapid cycles, which ultimately results in smaller 
constant incremental releases of software. As well as this major advantage, 
\citet{dawson09} states the following advantages of using the agile development 
model:

\begin{itemize}
  \item The methodology surrounds the concept of regular face-to-face meetings 
  as opposed to in-depth documentation
  \item Utilises a close working relationship between the client and the 
  developers, thus providing continuous delivery of useful software
  \item Uses shorter, iterative time scales (usually weeks rather than months 
  or years), which results in working software being delivered frequently
  \item Easily able to change the requirements at any stage (however late the 
  changes are)
\end{itemize}


\subsection{Disadvantages}

However many of the disadvantages of agile development model are surrounded by 
the lack of a rigid documentation, as \citet{dawson09} also suggests the following 
disadvantages:

\begin{itemize}
  \item There is often a lack of emphasis on necessary documentation (user 
  documentation, design documentation etc.), which is normally skipped to save 
  time
  \item The uncertainty of a specification may lead to poor code and/or 
  structure
  \item The project can become confused if the original specification is not 
  clear from the start
  \item Some software deliverables can be difficult to allocate the correct 
  amount of resources (time, effort etc.) at the start of the project
\end{itemize}



% Rapid Application Development overview 
\newpage
\section{Rapid Application Development}
The Rapid Application Development (RAD) model is an extension to the 
incremental development methodology. The RAD model states that all requirements
should be treated as mini projects, and that they should be completed in 
parallel. Each of the mini projects are ran like a normal project, and hence 
time scales need to be adhered to \citep{istqb10}.

Upon completion of the mini project, the customer is able to review the output,
and provide value feedback regarding to the delivery and the requirements. RAD 
will follow a somewhat simpler primary path, allowing for business modelling, 
data modelling, process modelling, application generation, testing and turnover
\citep{istqb10}.


\subsection{Advantages}
\citet{istqb10} states that there are many advantages of adopting the RAD model
within a team:

\begin{itemize}
  \item A reduced development time, due to the fact that the business modelling
  and data modelling processes should cover all aspects
  \item The combination of Data modelling and Process modelling should allow 
  for the increased ability to reuse components
  \item Reviews of delivered outputs are constantly reviewed by the customer, 
  allows for early feedback to be gained
  \item Parts of the system are integrated at an earlier stage, which allows 
  for fewer integration issues towards the end of the project
\end{itemize}


\subsection{Disadvantages}
However, \citet{istqb10} also states that there can be disadvantages of 
adopting the RAD model within a team:

\begin{itemize}
  \item There is a high dependency upon an overall strong team and strong 
  individual performances for identifying business requirements
  \item The model will only work for systems that can be modularised
  \item The model assumes that the team members are highly skills designers and
  developers, with an even higher dependency upon modelling skills
\end{itemize}



% Final chapter summary
\section{Summary}
In order to achieve the best possible product, it is clearly evident that the 
project should be developed utilising a feature driven approach. This will 
allow for any revisions, modifications, and changes to be considered and 
implemented with as little delay as possible, as well as little impact upon the
rest of the project.

The projects requirements are not set directly by an external client, and hence
it is possible for the requirements to be changed. It is because of these 
uncertainties that an agile development methodology would be best adopted by 
this project.

This methodology will not only allow for the requirements to change, but can 
allow for substantial research to be able to take place upon new topic areas if
needed. Agile development methodologies allow for multiple releases of 
software, which fundamentally means that the team is able to use prototyping 
techniques to find the best outcome to a given problem.
