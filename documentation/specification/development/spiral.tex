\section{Spiral}

A common feature found in the waterfall model is that all requirements are 
stated at the start of the project. It is these requirements that will form the
basis of all work, along with any project planning \citep{cadle10}.

The spiral model forms its basis around iteration and prototyping to try to 
explore the requirements and develop the solution. During each turn around the
spiral, a set of requirements are analysed and developed using prototyping 
\citep{cadle10}.


\subsection{Advantages}
\citet{cadle10} states that the spiral model houses a number of advantages, 
including:

\begin{itemize}
  \item A high amount of risk analysis is conducted, and thus risk is more 
  likely to be avoided
  \item The model allows for approval from clients, and large amounts of 
  documentation to be produced
  \item Software can start to be produced earlier, in comparison to the 
  waterfall methodology
  \item Additional functionality can be added on at any time during or after 
  the project
\end{itemize}


\subsection{Disadvantages}
The spiral model allows for a high level of control, without too much 
restriction. However \citet{cadle10} states that this can cause difficulties 
such as:

\begin{itemize}
  \item A thorough investigation into all of the requirements cannot be 
  achieved early, therefore some requirements (and their priorities) may get 
  completely missed
  \item The spiral model is based upon the clients knowing exactly what they 
  want, which is unlikely
  \item A risk analysis must be conducted, and requires highly specific 
  expertise to complete. IF a risk analysis is not completed, then the project
  may completely fail 
\end{itemize}
