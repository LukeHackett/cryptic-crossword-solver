\section{Waterfall}

One of the main aspects to the waterfall model is the fact that the project is 
expected to progress down the primary path \citep{cadle10}.

The waterfall model takes the major components of any project (requirements, 
design, implementation, testing and maintenance) and assigns each component a 
stage of its own. Each component is delivered as the flow down the primary path
is completed \citep{cadle10}.

The waterfall model also supports backtracking (i.e. reverting back to a 
previous deliverable). This allows for project managers to check that the 
project has not expanded its defined scope, and to also ensure that each 
deliverable flows into the next correctly. It also allows for slight 
modifications to be made, however making many large changes might affect the 
project in the long run \citep{cadle10}.


\subsection{Advantages}
\citet{cadle10} states that the waterfall model houses a number of advantages, 
including:

\begin{itemize}
  \item Provides a rigid project structure, that is easy to follow and review
  \item Deliverables are delivered in project order, one at a time
  \item Can work well for smaller projects, or for projects where by the 
  requirements will not change. 
\end{itemize}


\subsection{Disadvantages}
However \citet{cadle10} also goes on to state that the waterfall model houses a
number of potential problems, including:

\begin{itemize}
  \item Changes are difficult to implement the further a project is down its 
  primary path
  \item Large projects may not benefit from the rigid structure
  \item A working piece of software is not delivered until late into the 
  project
\end{itemize}
