\section{Problem Analysis}

An initial problem analysis has been conducted to ensure that the overall 
project remains focused upon the original problem.

This chapter will discuss key topics and will recommend that these topics are 
researched further to help support the project. The problem analysis will also 
define the problem in more detail, to help with understanding the project and 
its purpose.


\subsection{Defining the Environment}

Cryptic crosswords are a popular type of puzzles found in many parts of the 
world. Most commonwealth national newspapers will print cryptic crosswords of 
varying difficulty on a daily basis.

Cryptic crosswords are a unique style of crosswords, in which the answer to 
each given clue is a word puzzle. An answer can only be obtained if the cryptic
clue is read in the correct way. Often when the clue is surface read, the clue 
makes no sense at all. The challenge is to find a way in which the reading of 
the clue leads to a solution.


\subsection{Defining the Problem}

Many users can often become frustrated when a clue appears to be unsolvable. It
is the vast range of possible clues that often makes solving not only 
challenging but interesting as well.

Fundamentally, the overall aim of this project is to develop a piece of 
software that is able to solve any given type of cryptic crossword clue.

By using some form of natural language processing and one or more cryptic 
algorithms, it should be possible to generate an answer to a given clue.

Once a clue has been correctly ``guessed'' it can simply be returned to be 
user. It is the ``guessing'' of the answer that this project will primarily 
focus upon.


\subsection{Research Areas}

\subsubsection{Initial Survey}

Before undertaking the project an initial review was conducted. The review's 
objective was to determine the feasibility of the project as a whole. The 
review also covered whether or not the project has been completed before.

From the outlined background and problem information it is clear that cryptic 
crosswords are a popular form of entertainment. It is also clear that some 
clues are particularly difficult to solve, and users may often ask other people
for help in solving a given clue.


\subsubsection{Cryptic Crosswords}

A review of the national UK newspapers was conducted to determine whether or 
not there is a pattern in cryptic crosswords. Of all the newspaper's websites 
that were reviewed (The Guardian, The Times, The Independent and The Mirror) it 
was clear that the cryptic crosswords are the same style.

Each clue is categorised as being either `across' or `down' with its 
corresponding grid number. Each clue will also contain the number of letters 
the answer should be. An example is show below:

\begin{quote}
12. The seamstress's sensation? (4, 3, 7) =\textgreater  PINS AND NEEDLES
\end{quote}

The Guardian's website utilises web standard technologies such as HTML and CSS, 
and also provides an option to solve a clue. The Mirror's website follows a 
similar approach to the Guardian’s website; however solutions can only be 
obtained by dialing a premium telephone number.

The Times and the Independent both utilise a different approach and that is to 
serve a Java applet. Both Java applets allow the user to solve a clue should 
they get stuck. The Times provides puzzles as part of their paid subscription 
service.

All of the above newspapers publish cryptic crosswords upon a daily basis, with
the solutions to the crosswords appearing in the next day's newspaper.

Following from the crossword review, a second review into cryptic crossword 
solvers was undertaken. The objective of this review was to determine whether 
or not computerised cryptic crossword solvers exist. The three cryptic 
crossword solvers that were looked at were One Across, Crossword Tools and 
Cryptic Solver.

Each of the solvers manages to solve some clues with the same answers, with 
other clues providing a range of possible answers.

Crossword Tools \citep{crosswordtools} is a paid subscription based service, 
which allows users to enter a clue and a pattern. A pattern can contain part of
the answer or the number of letters the answer has. If multiple answers are 
available, they are displayed. An example is shown below:

\begin{quote}
Kind of dog (10) =\textgreater the answer is 10 letters long.

Kind of dog (?????????r) =\textgreater the answer is 10 letters long, final 
letter is `r'.
\end{quote}

Cryptic Solver \citep{crypticsolver} is a free service that offers the same 
functionality as Crossword Tools. Although Cryptic Solver does provide the 
correct answer, it does not necessarily provide the correct answer at the top 
of the list.

Finally One Across \citep{oneacross} provides all the same functionality as the
previous two solvers, along with a score. The score is linked to the number of 
people who have used the given answer (effectively it's a ratings system). One 
Across uniquely highlights how it has managed to come to the answer, showing 
the break downs of each sentence. As with Cryptic Solver, One Across is a free 
service that doesn't require a subscription.


\subsubsection{Natural Language Processing}

In order to correctly solve a clue, some form of natural language processing 
will be required. It is the natural language processing that will try to deduce 
the meaning of a clue. It is the meaning that can then be aligned with possible 
answers.

An example of natural language processing can be found within the One Across 
application. Given a clue (and a pattern) it will try to provide an accurate 
solution:

\begin{quote}
Spin broken shingle (7) =\textgreater  ENGLISH
\end{quote}

In order for the answer to be obtained, One Across will follow a natural 
language processing path and will provide it’s trace path. The trace path shows
how the clue has been broken down to get to the answer. The trace path for the 
above clue can be found below:

\begin{quote}
`spin' is the definition.

`broken' means to anagram `shingle' to get ENGLISH.

ENGLISH matches `spin' with confidence score 100\%.
\end{quote}

\subsubsection{Application Platform}

The existing products that have been discussed within this problem analysis 
have all been accessible via a browser. Although this is an acceptable 
platform, there could be a better platform that allows uses to utilise the 
technology easier.

As previously mentioned, most crosswords are designed for users who have a few 
minutes to spare on the move. As many people own a smartphone and/or a tablet, 
there may be a gap in the market for a high quality mobile cryptic crossword 
solver.

An in-depth review will need to be conducted in order to deduce the viability 
of this proposal.
