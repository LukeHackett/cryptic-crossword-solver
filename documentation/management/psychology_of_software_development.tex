%%% 
%% Project Management :: Psychology of Software Development 
%%%
\section{Psychology of Software Development}
\label{sec:psychology_of_software_development}

\citet{progpro86} states that the psychology of software development is often referred to as the psychology of programming which reflects its primary orientation to the coding phenomena that constitute rarely more that 15 percent of a large project effort. The need to share information on a software project by people is not enough and the communication needed to support this lack in many organisations. Larger teams and more members are required to produce systems which are rapidly growing in the current market in order to develop these systems quickly \citep{see81}. The ability to manage large software systems not only consists of empirical research but depends on the social and organisational aspects of the team. 

The team has looked at various aspects in regards to the structure of the organisation and found the most effective way to structure to the team is to have a democratic layout in order to allow all members sufficient participation. Not only have the members looked at the organisation structure but also the technical ability of each member in order to be served tasks which will push members to gain new knowledge or become an expert in a field. 

The path from a programmer to the end user can be very long and this is one of the many reasons in which the team took an iterative approach during development to ensure that there is a product by the end of the academic year allowing room for extra functionalities and enhancements to be added in the future.