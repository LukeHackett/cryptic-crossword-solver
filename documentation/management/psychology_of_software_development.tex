%%% 
%% Project Management :: Psychology of Software Development 
%%%
\section{Psychology of Software Development}
\label{sec:psychology_of_software_development}

\citet{progpro86} states that the psychology of software development is often
referred to as the psychology of programming which reflects its primary
orientation to the coding phenomena that constitutes rarely more that 15\% of a 
large project effort.

The need to share information on a software project by people is not enough and
the communication needed to support this is lacking in many organisations.
Larger teams and more members are required to produce systems which are rapidly
growing in the current market, in order to develop these systems quickly
\citep{see81}. The ability to manage large software systems not only consists of
empirical research but depends on the social and organisational aspects of the
team.

The team has looked at various aspects in regards to the structure of the
organisation and found the most effective way to structure to the team is to
have a democratic layout in order to allow all members sufficient participation.
Not only have the members looked at the organisation structure but also the
technical ability of each member in order to be served tasks which will push
members to gain new knowledge or become an expert in that field.

The path from a programmer to the end user can be long and this is one of the
reasons why the team took an iterative approach during development. This was to
ensure that there is a product by the end of the academic year, whilst also
allowing room for extra functionalities and enhancements to be added in the
future.

Although the initial Waterfall model suited well into the format of the academic
year the requirements produced reinforced that an iterative approach would be
better for development. This would ensure that all members were working
efficiently and were able to complete the various project requirements.

Other models were looked at and considers such as the agile development and
spiral methodologies, but the main reason why the iterative development
methodology was chosen was because it contained characteristics of the waterfall
model. Not only this, it managed to fit best into the team structure with the
minimal amount of time that was made available.
