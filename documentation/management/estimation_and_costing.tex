%%% 
%% Project Management :: Estimation and Costing 
%%% 
\section{Estimation and Costing} 
\label{sec:estimation_and_costing}

The costs of a software development project are primarily the costs of the
effort involved. The three main areas which correspond to the cost of a software
project include:

\begin{itemize}
  \item Hardware and software costs including maintenance
  \item Travel and training costs
  \item Effort costs (Costs of paying software engineers).
\end{itemize}

The main cost that will be discussed in this section is the effort costs as this
is the most adequate cost that can be calculated for this project.

In order to estimate the project costs the volume of the software must be
calculated. This can be achieved by calculating number of Source lines of Code
(SLOC) also calculated as Thousands of lines of Code (KLOC). The most common
estimating metric is the Constructive Cost Model (COCOMO) and the newer COCOMO 
II. Function points is another metrics that can be used to estimate the project.

COCOMO was originally developed by \citet{see81}. It has since been redeveloped
as COCOMO II \citep{cocomo2}.

The basic COCOMO formula is:

Effort Applied (Person Months)
\[
  E = a(KLOC)^{b}
\]

Development Time(Months)
\[
  DT = cE^{d}
\]

People Required
\[
  P = E/D
\]

\begin{table}[H]
  \centering
  \begin{tabular}{|l|l|l|l|l|}
    \hline
    \textbf{Mode} & \textbf{a} & \textbf{b} & \textbf{c} & \textbf{d} \\ \hline
    Organic       & 2.4        & 1.05       & 2.5        & 0.38       \\ \hline
    Semi Detached & 3.0        & 1.12       & 2.5        & 0.35       \\ \hline
    Embeded       & 3.6        & 1.20       & 2.5        & 0.32       \\ \hline
  \end{tabular}
  \caption{Basic COCOMO}
\end{table}

The cryptic solver best suits the Organic mode because the team is small 
consisting of four members who have experience working with requirements. 

On this basis and the time constraint factors the project can be estimated. 
The estimated number of lines of code SLOC has been set to 5 thousand. The
development time allocated for the project is 3.25 months which is the 98 days
calculated from the project plan. This has been considered to exclude time for
other academic deliverables and to mainly focus upon the actual development.
There fore according to \citet{cocomo2} the project can be estimated as
followed:

Effort Applied (Person Months)
\[
  E = 2.4(5)^{1.05} = 13
\]
Development Time (Months)
\[
  DT = 2.5(13)^{0.38} = 6.5 
\]
People Required
\[
  P = 13/6.5 = 2
\]

Now to evaluate this against the number of actual team members we can do the 
following.

Development Time (Months)
\[
  DT = 2.5(13)^{0.38}/2 = 3.25
\]
People Required
\[
  P = 13/3.25 = 4
\]

Upon completion of the project the COCOMO model was applied to see how closely 
the original estimation was made. In total there are 7731 SLOC and therefore 
applying the same algorithms as before produced the following results:

Effort Applied (Person Months)
\[
  E = 2.4(8)^{1.05} = 21
\]

Development Time (Months)
\[
  DT = 2.5(15)^{0.38}/2 = 4
\]

People Required
\[
  P = 21/4 = 5
\]
