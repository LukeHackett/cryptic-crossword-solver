%%% 
%% Project Management :: Ethics and Professionalism 
%%% 
\section{Ethics and Professionalism} 
\label{sec:ethics_and_professionalism}

In software engineering ethics is the study of right and wrong in relation to
human actions \citep{profissues01}. The participants of the group project have
looked at all aspects of ethical thinking as well as professionalism throughout
the course of the project.

The group members adopted ethical codes in order to demonstrate their
professionalism within the academic institution and to re-enforce these skills
in the working industry. The seven main functions of ethical thinking that were
looked at were:

\begin{itemize}
  \item Professionalism
  \item Protection of group interests
  \item Etiquette and inspiration
  \item Education
  \item Enforcement
  \item Principles, ideals and rules
  \item Rights - obligations and duties as well as the rights of members and the
        obligation of the professional body to its members. 
\end{itemize}


%%% 
%% Project Management :: Ethics and Professionalism 
%%                    :: Professional Codes of Conduct
%%% 
\subsection{Professional Codes of Conduct}

As students of an computing discipline each participant is a member of the 
British Computing Society (BCS). As a member of this professional body each 
member adheres to the rules which cover four main areas:

\begin{itemize}
  \item Public interest
  \item Professional competence and integrity
  \item Duty to relevant Authority
  \item Duty to the profession
\end{itemize}

\begin{flushright}
  {\footnotesize \citet{bcs14}}
\end{flushright}

For the purpose of the project the duties of each member of the team has been to
ensure that the all work, research has been conceptual, empirical and fully
technical.

As well as sticking to the BCS code of conduct the team members are aware of the
Student Handbook at the University of Huddersfield and are obligated to follow
these rules and regulations set out in this handbook.


%%% 
%% Project Management :: Ethics and Professionalism 
%%                    :: Intellectual Property Rights
%%% 
\subsection{Intellectual Property Rights}

The cryptic crossword solver is an academic application which is formally a
prototype solution to a real world problem. The decision has been taken by each
team member that there will be no claims for rights to the application and its
documents and after the marks have been awarded from the examining board. The
final project (documentation and the software product) application is to be
published as an open source project.

The project does not contain any confidential information which are internal to
the software engineering team and external to the clients. The members of the
group have also adhered to the Copyright, Patents and Designs Act 1988 and
ensured that all material used within the project is of their own and permission
has been granted from the relevant authorities such as the Guardian newspaper to
ensure that there are no conflicts in the code, design and documentation of the
project.


%%% 
%% Project Management :: Ethics and Professionalism :: Computer Misuse
%%% 
\subsection{Computer Misuse}

The members of the team are aware of the Computer Misuse Act 1990 and have
followed the correct procedures to ensure that all material that has been
generated, has been done in a lawful manner.

To provision the project the credentials for access to the Helios server and the
Amazon EC2 server have been maintained by the all users ensuring not to share
this information to anyone other than the team members.


The Guardian Newspaper was contacted with regards to their data upon their 
website by the team, to ensure that their data could be used within the project.
A copy of the email, and the response can be found upon the next page.

%% Guardian Response
\newpage
{
\footnotesize
\begin{verbatim}
From: Permissions Syndication <permissions.syndication@theguardian.com>
Sent: 13 November 2013 10:39
To: S.Leader U0955187
Subject: Re: Cryptic Crossword Data for an Academic Project
 
Dear Stuart,
 
Provided that the use is strictly non-commercial and educational I am happy for 
you to use Guardian articles as source material for you project free of charge. 
 
Please can you ensure that the story is credited to Guardian News & Media Ltd 
(year of publication) 

Best regards,

Helen
 
Helen Wilson
Content Sales Manager
Syndication
web | print | tablet | mobile
 
T: +44 (0) 20 3353 2367
M: +44 (0) 7717 807 973
Guardian News & Media Ltd, Kings Place, 90 York Way, London, N1 9GU

-------------------------------------------------------------------------------

On 12 November 2013 22:17, S.Leader U0955187 <U0955187@unimail.hud.ac.uk> wrote:

Hello,

I am currently a student of Software Engineering at the University of
Huddersfield. As part of my studies I am required to complete a group project,
and for this we have chosen to design and produce a piece of software that will
attempt to provide the correct solutions to given cryptic crossword clues.

As part of the development process, we expect to need an amount of training
data; that is collections of cryptic clues with their corresponding answers. We
would like to ask the Guardian's permission to use a few of the clues and
answers that are publicly available on the crossword section of the Guardian's
website. This would help us tremendously in the success of our academic project.

Please let me know if you would like clarification of any aspect of our project.

Kind regards,

Stuart Leader - u0955187
MEng Software Engineering

--------------------------------------------------------------------------
\end{verbatim}
}
