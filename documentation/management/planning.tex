%%% 
%% Project Management :: Planning 
%%% 
\section{Planning}
\label{sec:planning}

This section contains the project plans which have been developed in Microsoft
Project 2010. There are three sub sections which define the initial, interim and
end progress of the project. \citet{planning05} refers to software engineering
as two domains:

\begin{enumerate}
  \item Student Systems - Programs that people build to illustrate something or 
        for hobby.
  \item Industrial Strength Software - Software that can lead to significant 
        direct or indirect loss.
\end{enumerate}

A software project that is planned thoroughly is a more successful project than
a project that is developed through extreme programming. The Cryptic Crossword
project initially started through the Waterfall life cycle model and gradually
grew to become an Iterative model. Therefore the project plans had to be updated
constantly to reflect the change in which the project was being undertaken.

Although the overall project is developed in the Waterfall model due to the time
constraints applied by University deadlines the project consists of an iterative
development outside of the deliverables which are to be produced for academic
purposes.


\begin{landscape}

\subsection{Timeline}

\begin{figure}[H]
  \centering
  \includegraphics[width=\linewidth]{images/timeline1.png}
  \caption{Project Timeline (21$^s$$^t$ October 2013)}
  \label{fig:timeline1}
\end{figure}

Figure ~\ref{fig:timeline1} shows the initial time scale that was drawn up while
planning the project. This was produced to show the major deliverables and their
dates. In the next section the Gantt Chart that was produced with all the
initial tasks to be completed are shown.

\newpage 
\subsection{Initial Gantt Chart}

\begin{figure}[H]
  \centering
  \includegraphics[width=\linewidth]{images/gant_chart1_term1.png}
  \caption{Gantt Chart for Term one (21$^s$$^t$ October 2013)}
  \label{fig:ganttinitialterm1}
\end{figure}

\begin{figure}[H]
  \centering
  \includegraphics[width=\linewidth]{images/gant_chart1_term2.png}
  \caption{Gantt Chart for Term two (21$^s$$^t$ October 2013)}
  \label{fig:ganttinitialterm2}
\end{figure}

\subsection{Timeline Re-Visited}

\begin{figure}[H]
  \centering
  \includegraphics[width=\linewidth]{images/timeline2.png}
  \caption{Project Timeline (30$^t$$^h$ January 2014)}
  \label{fig:timeline2}
\end{figure}


\newpage 
\subsection{Interim Gantt Chart}

\begin{figure}[H]
  \centering
  \includegraphics[scale=0.35]{images/gant_chart_interim_term1.png}
  \caption{Gantt Chart for Term one (30$^t$$^h$ January 2014)}
  \label{fig:ganttinterimterm1}
\end{figure}

\begin{figure}[H]
  \centering
  \includegraphics[width=\linewidth]{images/gant_chart_interim_term2.png}
  \caption{Gantt Chart for Term two (30$^t$$^h$ January 2014)}
  \label{fig:ganttinterimterm2}
\end{figure}



\subsection{Final Gantt Chart}

\begin{figure}[H]
  \centering
  \includegraphics[width=\linewidth]{images/gant_chart_final_overview.png}
  \caption{Final Gantt Chart of Project Summary (7$^t$$^h$ May 2014)}
  \label{fig:ganttfinaloverview}
\end{figure}

\begin{figure}[H]
  \centering
  \includegraphics[width=\linewidth]{images/gant_chart_final_term2.png}
  \caption{Final Gantt Chart for Term two (7$^t$$^h$ May 2014)}
  \label{fig:ganttfinalterm2}
\end{figure}

\end{landscape}
