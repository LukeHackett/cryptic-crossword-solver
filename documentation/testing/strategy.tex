%%%
%% Testing :: Test Strategy
%%%
\section{Test Strategy}
\label{sec:test_strategy}

Within this section the proposed test strategy used through the testing stage of 
the project will be outlined. A test strategy describes the various approaches 
to testing the development phase of a software development project. 


%%%
%% Testing :: Test Strategy :: Approaches
%%%
\subsection{Approaches}
\label{sub:approaches}

The testing approaches that will be used during the testing of this project will
be unit testing, integration testing, system testing and requirements testing. 
Each of these types of tests will individually focus upon one aspect of testing,
but when combined together will form a larger, more complete test approach.

Due to the size of the group, the tests will be conducted by those who will have
been developing the software --- ideally these two tasks would be completed by 
separate teams.

The unit testing will be completed using the JUnit test framework and will test
the various `back end' classes that make up the system. The unit tests will also
cover many of the integration tests, which ensures that classes when combined 
together or functioning correctly. 

A number of full system walkthoughs will also be conducted manually. This will 
ensure that the user interface has been correctly developed, as well as ensuring
that the user interface is as fluid as possible. 

Finally the software product will undergo requirements testing. This will 
compare the original specification against the software product, and will 
provide evidence of the feature being implemented. If a feature has not been 
implemented, a reason as to why not will be discussed.]


%%%
%% Testing :: Test Strategy :: Environments & Tools
%%%
\subsection{Environments \& Tools}
\label{sub:environments_and_tools}

The test environment and tools are similar to those required within the 
development phase. To clarify these environments and tools are:

\begin{itemize}
  \item Java 7
  \item Apache Tomcat 7
\end{itemize}

As well as the above fundamental tools, some additional testing tools and 
libraries will be required, and are outlined below:

\begin{itemize}
  \item JUnit 4 library
  \item Apache HTTPComponents library
  \item MySQL Database
\end{itemize}

The JUnit 4 library is used extensively throughout the testing phase to ensure
that the code is correctly working upon various levels.

The Apache HTTPComponents library provides low level access to the HTTP
protocol, which enables for responses to be tested against their subsequent
requests. For example if the HTTP return type header was XML does the system
actually deliver XML through the HTTP protocol.

Finally the MySQL database stores a number of cryptic clues, their solutions and
the type of clue it is. This will enable the system to be tested with real data
and thus test to ensure the algorithms are correctly generating the correct 
results. It will also help to test that the results are being `ranked' 
appropriately using the confidence rating.


%%%
%% Testing :: Test Strategy :: Test Schedule
%%%
\subsection{Test Schedule}
\label{sub:test_schedule}

In order to prevent the testing phase becoming too long, a time limit of 20\% 
per phase has been set. This means that up to 20\% of the allocated time for a 
given phase will be devoted to testing that phase.

It must be said that testing will be conducted alongside development, and 
therefore many of the major and common aspects of the system should have had 
some form of (basic) testing. The 20\% time limit has been put in place to 
ensure that there is some time set a side for testing.

The time allocation will also be used to help schedule how long a development 
phase should last, and thus in directly controls how much testing will need to
be conducted.


%%%
%% Testing :: Test Strategy :: Test Priorities
%%%
\subsection{Test Priorities}
\label{sub:test_priorities}

The test priorities will be in line with the MoSCoW analysis. This will mean 
that requirements that fall under the `Must' and `Should' categories will by 
default have a higher testing priority over `Could' categories.

The project is focusing upon the `back end' functionality as much as possible, 
and therefore there will be no automated user interface tests. This also means
that `back end' tests will have a higher priority over user interface tests. 
However it must be stated that the reduction in priority does not mean that 
there will be a lack in quality.


%%%
%% Testing :: Test Strategy :: Test Groups
%%%
\subsection{Test Groups}
\label{sub:test_groups}

In order to ensure that the product is fit for it's intended purpose it was 
proposed that a number of quality circles would be run. These circles known as 
`test groups', would identify issues within the product whilst also providing an
`idealistic solution'. The solutions are branded as idealistic as they may not 
always be feasible with the resources that are available.

Although one or more test groups would have provided various levels of feedback
it was decided that the current levels of resources did not permit a full 
evaluation of the results. However if more time (and resources) were available 
the group would undertake a test group as part of the testing cycle.


%%%
%% Testing :: Test Strategy :: Test Reports
%%%
\subsection{Test Reports}
\label{sub:test_reports}

For each of the highlighted testing areas --- unit testing, integration testing,
system testing and requirements testing --- a dedicated subsection will be 
made available as part of the testing section that makes up the written report.

Each of the subsections will highlight the specifics of type of test, and 
present the results.

It must be stated that the reports will only show the latest phase of tests. 
Prehistoric tests will not be presented to avoid repetition and outdated
information.
